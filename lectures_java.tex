\documentclass{article}
\usepackage[utf8]{inputenc}
\usepackage[T2A]{fontenc}
\usepackage[russian]{babel}
\usepackage{hyperref}
\usepackage{amsthm}
\usepackage{mathtools}
\usepackage{listings}
\usepackage{xcolor}
\newtheorem{theorem}{Теорема}
\newtheorem{corollary}{Следствие}[theorem]
\newtheorem{lemma}[theorem]{Лемма}

\hypersetup{
    colorlinks=true,
    linkcolor=blue,
    filecolor=magenta,      
    urlcolor=cyan,
    pdftitle={Java Lectures},
    pdfpagemode=FullScreen,
}
\definecolor{codegreen}{rgb}{0,0.6,0}
\definecolor{codegray}{rgb}{0.5,0.5,0.5}
\definecolor{codepurple}{rgb}{0.58,0,0.82}
\definecolor{backcolour}{rgb}{0.95,0.95,0.92}

\lstdefinestyle{mystyle}{
    backgroundcolor=\color{backcolour},   
    commentstyle=\color{codegreen},
    keywordstyle=\color{magenta},
    numberstyle=\tiny\color{codegray},
    stringstyle=\color{codepurple},
    basicstyle=\ttfamily\footnotesize,
    breakatwhitespace=false,         
    breaklines=true,                 
    captionpos=b,                    
    numbers=left,                    
    numbersep=5pt,                  
    showspaces=false,                
    showstringspaces=false,
    showtabs=false,                  
    tabsize=2
}

\lstset{style=mystyle}

\title{Конспект лекций по java}
\author{Александр Титилин}
\date{}
\begin{document}
    \maketitle
    \tableofcontents
    \section{Шапочка}
    \begin{lstlisting}[language=Java] 
        public class Main {
            public static void main(String[] args){
                write here
            }
        }
    \end{lstlisting} 
    \section{Hello World}
    \begin{lstlisting}[language=Java] 
        System.out.print("Hello World"); 
        System.out.println();
        System.out.print("How are you?");
    \end{lstlisting} 

    print - вывести в консоль без переноса строки.

    println - вывести в консоль c переносом строки.
    \section{Целочисленный тип данный int}

    Создание переменнной с именем varibleName c типом  dataType

    \begin{lstlisting}[language=Java] 
    dataType varibleName; 
    \end{lstlisting} 
    Переменная sum  с типом данных int;

    \begin{lstlisting}[language=Java] 
    int sum; 
    \end{lstlisting} 
    Положим в sum число 12.
    \begin{lstlisting}[language=Java] 
    sum = 12; 
    \end{lstlisting} 
    Создаем перменную sum и сразу присваиваем ей значение 12
    \begin{lstlisting}[language=Java] 
    int sum = 12; 
    \end{lstlisting} 
    Описываем несколько переменных и некоторые инициализируем
    \begin{lstlisting}[language=Java] 
    int a , b , c = 5, d , e = 1234; 
    \end{lstlisting} 
    Вывод значение переменной sum на экран.
    \begin{lstlisting}[language=Java] 
    System.out.println(sum); 
    \end{lstlisting} 
    Выводить можем и значения выражений.
    \begin{lstlisting}[language=Java] 
    System.out.println(sum + 129291);
    \end{lstlisting} 
    \begin{lstlisting}[language=Java] 
    System.out.println("next after sum" + sum + 1);
    \end{lstlisting} 
    Напишет "... 121"
    \begin{lstlisting}[language=Java] 
    System.out.println("next after sum" + ( sum + 1 ));
    \end{lstlisting} 
    Напишет "... 13"

    int - целое число, размером 4 байта.
    Имена переменных начинаются с маленькой буквы,
    состоящие из нескольких слов используют camelCase
    (countOfEvenDigits).
    \section{Ввод с клавиатуры}
    Для ввода с клавиатуры нужен объект Scaner из шапки с сайта школы.

    \subsection{ Получение целого числа с клавиатуры. }
    \begin{lstlisting}[language=Java] 
    sum = in.nextInt() 
    \end{lstlisting} 
    Так тоже можно
    \begin{lstlisting}[language=Java] 
    int second = in.nextInt();
    \end{lstlisting} 
    \section{Действия с целыми числами}
    Арифметические действия как везде ($*,+,-,/$,\%), деление целочисленное.
    Побитовый сдвиг влево
    \begin{lstlisting}[language=Java] 
    a << b;
    \end{lstlisting} 
    Побитовый сдвиг вправо.
    \begin{lstlisting}[language=Java] 
    a >> b;
    \end{lstlisting} 
    Знаковый побитовый сдвиг вправо.
    \begin{lstlisting}[language=Java] 
    a >>> b; 
    \end{lstlisting} 
    \subsection{Замечание. Деление отрицательных чисел}

    \begin{tabular}
        {c | c | c | c}
        Делимое & Делитель & Целое & Остаток \\
        \hline
        23 & 5 & 4 & 3\\
        \hline
        -23 & 5 & -4 & -3\\
        \hline
        23 & -5 & 4 & 3 \\
        \hline
        -23 & -5 & 4 & -3\\
        \hline
    \end{tabular}

       В питоне и математике не так. 
    \section{Целочисленные типы данных в java}

    \begin{tabular}
        {c | c | c | c}
        Тип Данных  & Размер ячейки & Размер в битах & Диапозон \\
        \hline
        byte & 1 байт & 8 & $-128 \dots 127$ \\
        short & 2 байта & 16 & $-32768 \dots +32767$ \\
        int & 4 байта & 32 & $-2^{31} + \dots  2^{31} - 1$ \\
        long & 8 байт & 64 & $-2^{63} + \dots + 2^{63} - 1$ \\
        \hline
    \end{tabular}
    \section{Ерунда про двоичное представление, про знак и прочее}
    Cуть в том, что старший разряд в двоичном представлении это знак.
    Про дополнительный двоичный код.
    Пока джавы нет идут байки про работу компа.
    \section{15.09.2022}
    \subsection{Вещественный тип данных}
    Вещественные типы данных в java - float(4 байта) и double(8 байт). Если можно не использовать вещественные числа, то их не надо использовать. Прикол про $0.1 + 0.2 \neq 0.3$. Любые числа с вещественными числами будут приближенными.
    \begin{lstlisting}[language=Java] 
    float f = 1.7;
    double d = 1.7;
    \end{lstlisting} 
    Первое работать не будет.
    \subsection{Ввод}
    \begin{lstlisting}[language=Java] 
    double x = in.nextDouble() ;
    \end{lstlisting} 
    \subsection{Вывод на экран}
    \begin{lstlisting}[language=Java] 
    out.printf("%.3f",x); 
    \end{lstlisting}  
    Вывод вещественного числа x с 3 числами после запятой.
    \begin{lstlisting}[language=Java] 
    out.printf("Answer: %d %.2f\n",a,x);
    \end{lstlisting} 
    Вывод слова, целого числа a в десятичном представлении и вещественного числа x с двумя числами после запятой и перевод строки.
    \subsection{Операции с вещесвенными числами}
    Арифметические как в целых, кроме деления.
    \begin{lstlisting}[language=Java] 
    int b = 23,c=5;
    double y = b/c;
    \end{lstlisting} 
    Y будет равен четырем. Надо делать так
    \begin{lstlisting}[language=Java] 
    double y =  (double)b / c;
    \end{lstlisting} 
    Надо какое нибудь число привести к вещественным.
    \subsection{Приведения действительных чисел к целым}
    \begin{lstlisting}[language=Java] 
    int g = (int) x;
    \end{lstlisting} 
    \section{Математические функции в java.}
    Все такие функции лежат в библиотеке Math, подключать не надо.

    \begin{tabular}
        {c | c |}
        Math.abs(x) & $\mid x \mid$ \\
        Math.sqrt(x) & $\sqrt{x} $ \\
        Math.sin(x) & $\sin{x}$ \\
        Math.cos(x) & $\cos{x}$ \\
        Math.tan(x) & $\tan{x}$ \\
    \end{tabular}
    \subsection{Символьный тип данных char}
    \begin{lstlisting}[language=Java] 
    char c = 'F';
    \end{lstlisting} 
    Представляет собой целое беззнаковое число, занимает 2 байта.
    \subsection{Ввод с клавиатуры}
    \begin{lstlisting}[language=Java] 
    char h = (char) System.in.read();
    \end{lstlisting} 
    При этом компилятор ругнется. Нужно использовать альтернативную шапочку
    \begin{lstlisting}[language=Java] 
    import java.io.IOException
        public class Main {
            public static void main(String[] args) throws IOException 
            {
                write here
            }
        }
    \end{lstlisting} 
\section{\date{22.09.2022}}
\subsection{Логический тип данных}
\begin{lstlisting}[language=Java] 
boolean b = true; 
\end{lstlisting} 
\subsection{Операции сравнения}
\begin{tabular}
    {c | c}
    Математика & Java \\
    \hline
    $>$ & >\\
    $<$ & >\\
    $=$ & == \\
    $\ge $ & >=\\
    $\le $ & <=\\
    $\neq$ & !=\\
\end{tabular}
\begin{lstlisting}[language=Java] 
boolean c = 3 > 5;
out.print(c);
\end{lstlisting} 
Cчитывать boolean нельзя.
\subsection{Логические операции}
\begin{tabular}
    {|c | c | c |}
    \hline
    Операция & Обозначение & Смысл  \\
    \hline
    Не (инверсия) & ! & Меняет логическое значение на противоположное \\
    \hline
    И (коньюнкция) & $\&\&$ & Истина, если оба операнда истина \\
    \hline
    Или (дизъюнкция) & || & Истина, если хотя бы один операнд истина \\
    \hline
    Xor & $\land$ & Истинна если операнды разные \\
    \hline
\end{tabular}
В java нельзя использовать двойные сравнения.
\subsection{Условный оператор}
\begin{lstlisting}[language=Java] 
if (cond){
    operator-Yes;
} 
\end{lstlisting} 
\begin{lstlisting}[language=Java] 
if (cond){
    operator-Yes;
}
else {
    operator-No;
}
\end{lstlisting} 
Примеры.
\begin{lstlisting}[language=Java] 
if (x > 10) {
    System.out.println("Too much");
} 
\end{lstlisting} 
\begin{lstlisting}[language=Java] 
    if (x > 10){
        System.out.println("Too much");
    } 
    else {
        System.out.println("Good");
    }
\end{lstlisting} 
\subsection{Область видимости переменных.}
Создание переменной внутри блока 
\begin{lstlisting}[language=Java] 
{
    int a = 10;
} 
\end{lstlisting} 
Внутри скобок использовать можно, снаружи нет.
\section{Цикл с счетчиком \underline{for}}
Цикл повторяющаяся последовательность действий, которые называются телом цикла. 
\begin{lstlisting}[language=Java] 
for(start values; condition;change counter) {
    body
}
\end{lstlisting} 
В блоке начальных значений, можно описывать переменнные. В блоке изменения счетчика можно менять несколько начальных значений.
\begin{lstlisting}[language=Java] 
for(int i = 0,k,j = ; i < 10 && j < 1000;i++,j+10) 
\end{lstlisting} 
Эти переменнные пропадут, после окончания цикла.
\begin{lstlisting}[language=Java] 
for(int i = 5; i <= 8; i ++) 
\end{lstlisting} 
Выведет 
\begin{lstlisting}[language=Java] 
5
6
7
8
\end{lstlisting} 
\subsection{Задачка}
Дано натуральное число, нужно вывести первые n четных чисел.
\subsubsection{Первый cпособ, формулкой}
\begin{lstlisting}[language=Java] 
for(int i = 0; i < n ; i++){
    out.println(2*(i + 1));
}
\end{lstlisting} 
\subsubsection{Второй способ, дополнительная переменная}
\begin{lstlisting}[language=Java] 
int a = 2
for(int i = 0; i < n; i++){
    out.println(a);
    a += 2;
}
\end{lstlisting} 
Тоже самое, но короче
\begin{lstlisting}[language=Java] 
for (int i = 0,a = 2; i < n;i++,a+=2) {
    out.println(a);
}
\end{lstlisting} 
\subsection{Задача}
Дано число \underline{n} (количество элементов последовательности), после этого даны n чисел.
Надо найти $\sum$ четных элементов данной последовательности
\begin{lstlisting}[language=Java] 
int n = in.nextInt(); 
int sum = 0;
for (int i = 0,a; i < n ; i++){
   a = in.nextInt();
   if (a % 2 == 0){
        sum +=a;
   }
}
out.println(sum);
\end{lstlisting} 
\subsection{Правило}
 
Начальные значения переменных (суммы, количества) нужно задавать
непосредственно перед тем циклом, в котором они изменяются.
\end{document}
