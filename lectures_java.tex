\documentclass{scrartcl}
\usepackage[utf8]{inputenc}
\usepackage[T2A]{fontenc}
\usepackage[russian]{babel}
\usepackage{amsthm}
\usepackage{mathtools}
\usepackage{listings}
\usepackage{xcolor}
\usepackage{geometry}
\newtheorem{task}{Задача}
\definecolor{codegreen}{rgb}{0,0.6,0}
\definecolor{codegray}{rgb}{0.5,0.5,0.5}
\definecolor{codepurple}{rgb}{0.58,0,0.82}
\definecolor{backcolour}{rgb}{0.95,0.95,0.92}

\lstdefinestyle{mystyle}{
	backgroundcolor=\color{backcolour},
	commentstyle=\color{codegreen},
	keywordstyle=\color{magenta},
	numberstyle=\tiny\color{codegray},
	stringstyle=\color{codepurple},
	basicstyle=\ttfamily\footnotesize,
	breakatwhitespace=false,
	breaklines=true,
	captionpos=b,
	showspaces=false,
	showstringspaces=false,
	showtabs=false,
	tabsize=2
}

\lstset{style=mystyle}

\title{Конспект лекций по java}
\author{Александр Титилин}
\date{}
\begin{document}
\maketitle
\tableofcontents
\section{Шапочка}
\begin{lstlisting}[language=Java] 
        public class Main {
            public static void main(String[] args){
                write here
            }
        }
    \end{lstlisting}
\section{Hello World}
\begin{lstlisting}[language=Java] 
        System.out.print("Hello World"); 
        System.out.println();
        System.out.print("How are you?");
    \end{lstlisting}

print - вывести в консоль без переноса строки.

println - вывести в консоль c переносом строки.
\section{Целочисленный тип данный int}

Создание переменнной с именем varibleName c типом  dataType

\begin{lstlisting}[language=Java] 
    dataType varibleName; 
    \end{lstlisting}
Переменная sum  с типом данных int;

\begin{lstlisting}[language=Java] 
    int sum; 
    \end{lstlisting}
Положим в sum число 12.
\begin{lstlisting}[language=Java] 
    sum = 12; 
    \end{lstlisting}
Создаем перменную sum и сразу присваиваем ей значение 12
\begin{lstlisting}[language=Java] 
    int sum = 12; 
    \end{lstlisting}
Описываем несколько переменных и некоторые инициализируем
\begin{lstlisting}[language=Java] 
    int a , b , c = 5, d , e = 1234; 
    \end{lstlisting}
Вывод значение переменной sum на экран.
\begin{lstlisting}[language=Java] 
    System.out.println(sum); 
    \end{lstlisting}
Выводить можем и значения выражений.
\begin{lstlisting}[language=Java] 
    System.out.println(sum + 129291);
    \end{lstlisting}
\begin{lstlisting}[language=Java] 
    System.out.println("next after sum" + sum + 1);
    \end{lstlisting}
Напишет "... 121"
\begin{lstlisting}[language=Java] 
    System.out.println("next after sum" + ( sum + 1 ));
    \end{lstlisting}
Напишет "... 13"

int - целое число, размером 4 байта.
Имена переменных начинаются с маленькой буквы,
состоящие из нескольких слов используют camelCase
(countOfEvenDigits).
\section{Ввод с клавиатуры}
Для ввода с клавиатуры нужен объект Scaner из шапки с сайта школы.

\subsection{ Получение целого числа с клавиатуры. }
\begin{lstlisting}[language=Java] 
    sum = in.nextInt() 
    \end{lstlisting}
Так тоже можно
\begin{lstlisting}[language=Java] 
    int second = in.nextInt();
    \end{lstlisting}
\section{Действия с целыми числами}
Арифметические действия как везде ($*,+,-,/$,\%), деление целочисленное.
Побитовый сдвиг влево
\begin{lstlisting}[language=Java] 
    a << b;
    \end{lstlisting}
Побитовый сдвиг вправо.
\begin{lstlisting}[language=Java] 
    a >> b;
    \end{lstlisting}
Знаковый побитовый сдвиг вправо.
\begin{lstlisting}[language=Java] 
    a >>> b; 
    \end{lstlisting}
\subsection{Замечание. Деление отрицательных чисел}

\begin{tabular}
	{c | c | c | c}
	Делимое & Делитель & Целое & Остаток \\
	\hline
	23      & 5        & 4     & 3       \\
	\hline
	-23     & 5        & -4    & -3      \\
	\hline
	23      & -5       & 4     & 3       \\
	\hline
	-23     & -5       & 4     & -3      \\
	\hline
\end{tabular}

В питоне и математике не так.
\section{Целочисленные типы данных в java}

\begin{tabular}
	{c | c | c | c}
	Тип Данных & Размер ячейки & Размер в битах & Диапозон                       \\
	\hline
	byte       & 1 байт        & 8              & $-128 \dots 127$               \\
	short      & 2 байта       & 16             & $-32768 \dots +32767$          \\
	int        & 4 байта       & 32             & $-2^{31} + \dots  2^{31} - 1$  \\
	long       & 8 байт        & 64             & $-2^{63} + \dots + 2^{63} - 1$ \\
	\hline
\end{tabular}
\section{Ерунда про двоичное представление, про знак и прочее}
Cуть в том, что старший разряд в двоичном представлении это знак.
Про дополнительный двоичный код.
Пока джавы нет идут байки про работу компа.
\section{15.09.2022}
\subsection{Вещественный тип данных}
Вещественные типы данных в java - float(4 байта) и double(8 байт). Если можно не использовать вещественные числа, то их не надо использовать. Прикол про $0.1 + 0.2 \neq 0.3$. Любые числа с вещественными числами будут приближенными.
\begin{lstlisting}[language=Java] 
    float f = 1.7;
    double d = 1.7;
    \end{lstlisting}
Первое работать не будет.
\subsection{Ввод}
\begin{lstlisting}[language=Java] 
    double x = in.nextDouble() ;
    \end{lstlisting}
\subsection{Вывод на экран}
\begin{lstlisting}[language=Java] 
    out.printf("%.3f",x); 
    \end{lstlisting}
Вывод вещественного числа x с 3 числами после запятой.
\begin{lstlisting}[language=Java] 
    out.printf("Answer: %d %.2f\n",a,x);
    \end{lstlisting}
Вывод слова, целого числа a в десятичном представлении и вещественного числа x с двумя числами после запятой и перевод строки.
\subsection{Операции с вещесвенными числами}
Арифметические как в целых, кроме деления.
\begin{lstlisting}[language=Java] 
    int b = 23,c=5;
    double y = b/c;
    \end{lstlisting}
Y будет равен четырем. Надо делать так
\begin{lstlisting}[language=Java] 
    double y =  (double)b / c;
    \end{lstlisting}
Надо какое нибудь число привести к вещественным.
\subsection{Приведения действительных чисел к целым}
\begin{lstlisting}[language=Java] 
    int g = (int) x;
    \end{lstlisting}
\section{Математические функции в java.}
Все такие функции лежат в библиотеке Math, подключать не надо.

\begin{tabular}
	{c | c |}
	Math.abs(x)  & $\mid x \mid$ \\
	Math.sqrt(x) & $\sqrt{x} $   \\
	Math.sin(x)  & $\sin{x}$     \\
	Math.cos(x)  & $\cos{x}$     \\
	Math.tan(x)  & $\tan{x}$     \\
\end{tabular}
\subsection{Символьный тип данных char}
\begin{lstlisting}[language=Java] 
    char c = 'F';
    \end{lstlisting}
Представляет собой целое беззнаковое число, занимает 2 байта.
\subsection{Ввод с клавиатуры}
\begin{lstlisting}[language=Java] 
    char h = (char) System.in.read();
    \end{lstlisting}
При этом компилятор ругнется. Нужно использовать альтернативную шапочку
\begin{lstlisting}[language=Java] 
    import java.io.IOException
        public class Main {
            public static void main(String[] args) throws IOException 
            {
                write here
            }
        }
    \end{lstlisting}
\section{\date{22.09.2022}}
\subsection{Логический тип данных}
\begin{lstlisting}[language=Java] 
boolean b = true; 
\end{lstlisting}
\subsection{Операции сравнения}
\begin{tabular}
	{c | c}
	Математика & Java \\
	\hline
	$>$        & >    \\
	$<$        & >    \\
	$=$        & ==   \\
	$\ge $     & >=   \\
	$\le $     & <=   \\
	$\neq$     & !=   \\
\end{tabular}
\begin{lstlisting}[language=Java] 
boolean c = 3 > 5;
out.print(c);
\end{lstlisting}
Cчитывать boolean нельзя.
\subsection{Логические операции}
\begin{tabular}
	{|c | c | c |}
	\hline
	Операция         & Обозначение & Смысл                                         \\
	\hline
	Не (инверсия)    & !           & Меняет логическое значение на противоположное \\
	\hline
	И (коньюнкция)   & $\&\&$      & Истина, если оба операнда истина              \\
	\hline
	Или (дизъюнкция) & ||          & Истина, если хотя бы один операнд истина      \\
	\hline
	Xor              & $\land$     & Истинна если операнды разные                  \\
	\hline
\end{tabular}
В java нельзя использовать двойные сравнения.
\subsection{Условный оператор}
\begin{lstlisting}[language=Java] 
if (cond){
    operator-Yes;
} 
\end{lstlisting}
\begin{lstlisting}[language=Java] 
if (cond){
    operator-Yes;
}
else {
    operator-No;
}
\end{lstlisting}
Примеры.
\begin{lstlisting}[language=Java] 
if (x > 10) {
    System.out.println("Too much");
} 
\end{lstlisting}
\begin{lstlisting}[language=Java] 
    if (x > 10){
        System.out.println("Too much");
    } 
    else {
        System.out.println("Good");
    }
\end{lstlisting}
\subsection{Область видимости переменных.}
Создание переменной внутри блока
\begin{lstlisting}[language=Java] 
{
    int a = 10;
} 
\end{lstlisting}
Внутри скобок использовать можно, снаружи нет.
\section{Цикл с счетчиком \underline{for}}
Цикл повторяющаяся последовательность действий, которые называются телом цикла.
\begin{lstlisting}[language=Java] 
for(start values; condition;change counter) {
    body
}
\end{lstlisting}
В блоке начальных значений, можно описывать переменнные. В блоке изменения счетчика можно менять несколько начальных значений.
\begin{lstlisting}[language=Java] 
for(int i = 0,k,j = ; i < 10 && j < 1000;i++,j+10) 
\end{lstlisting}
Эти переменнные пропадут, после окончания цикла.
\begin{lstlisting}[language=Java] 
for(int i = 5; i <= 8; i ++) 
\end{lstlisting}
Выведет
\begin{lstlisting}[language=Java] 
5
6
7
8
\end{lstlisting}
\subsection{Задачка}
Дано натуральное число, нужно вывести первые n четных чисел.
\subsubsection{Первый cпособ, формулкой}
\begin{lstlisting}[language=Java] 
for(int i = 0; i < n ; i++){
    out.println(2*(i + 1));
}
\end{lstlisting}
\subsubsection{Второй способ, дополнительная переменная}
\begin{lstlisting}[language=Java] 
int a = 2
for(int i = 0; i < n; i++){
    out.println(a);
    a += 2;
}
\end{lstlisting}
Тоже самое, но короче
\begin{lstlisting}[language=Java] 
for (int i = 0,a = 2; i < n;i++,a+=2) {
    out.println(a);
}
\end{lstlisting}
\subsection{Задача}
Дано число \underline{n} (количество элементов последовательности), после этого даны n чисел.
Надо найти $\sum$ четных элементов данной последовательности
\begin{lstlisting}[language=Java] 
int n = in.nextInt(); 
int sum = 0;
for (int i = 0,a; i < n ; i++){
   a = in.nextInt();
   if (a % 2 == 0){
        sum +=a;
   }
}
out.println(sum);
\end{lstlisting}
\subsection{Правило}
Начальные значения переменных (суммы, количества) нужно задавать
непосредственно перед тем циклом, в котором они изменяются.
\section{16.10.2022}
\subsection{Поиск максимума.}
\begin{task}
	Дано число n. Затем ищем еще n целых чисел. Найти максимальный элемент.
\end{task}
\begin{lstlisting}[language=Java] 
int n = in.nextInt();
int max = in.nextInt();
int x;
for (int i = 0; i < n - 1;i++){
    x = in.nextInt();
    if (x > max){
        max = x;
    }
}
out.println(max);
    \end{lstlisting}
\begin{task}
	Максимум от функции. Элемент квадрат, которого максимален.
\end{task}
\begin{lstlisting}[language=Java] 
if (a * a > max*max) 
\end{lstlisting}
\begin{task}
	Максимум с условием. Максимальный четный элемент. Нельзя первый элемент в качестве начального элемента.
\end{task}
\subsubsection{}
Если известно ограничение на диапозон значений элементов. Тогда все просто, максимум равен минимуму диапазона+1 до цикла.

\subsubsection{}
Ограничения нет на диапазон. Сначала нужно найти первый элемент удовлетворяющий условию и его взять в качестве начального значения. Потом остальные сравниваем как обычно.
\begin{task}
	Найти максимальный четный элемент.
\end{task}
\begin{lstlisting}[language=Java] 
int max = 1;
for (int i = 0; i < n; i++){
    int a = in.nextInt();
    if (a % 2 == 0){
        if(max == 1 || a > max){
            max = a;
        }
    }
}
if (max == 1)
    out.println("NO");
else
    out.println(max);
\end{lstlisting}
\subsection{Цикл с условием}
\subsubsection{Цикл с предусловием.}
Сначала проверяет условие, потом делает тело цикла или выходит из цикла, потом проверяет условие.
\begin{lstlisting}[language=Java] 
while (cond) {
    body;
}
\end{lstlisting}
\subsubsection{Цикл с предусловием}
Сначала делает тело цикла, потом проверяет условие.
\begin{lstlisting}[language=Java] 
do {
    body;
}while(cond);
\end{lstlisting}
\subsection{Примеры}
\begin{task}
	Дана последовательность целых чисел, которая заканчивается числом 100. Найти сумму чисел, больше 20.
\end{task}
\subsubsection{Последний элемент не нужно учитывать.}
\begin{lstlisting}[language=Java] 
int a = in.nextInt();
int sum = 0;
while(a != 100){
    if (a > 20){
        sum += a;
    }
    a = in.nextInt();
}
out.println(sum);
\end{lstlisting}
\begin{lstlisting}[language=Java] 
int sum = 0; 
int a = in.nextInt();
if (a != 100){ 
do {
     a = in.nextInt();
    if (a > 20) {
        sum += a;
    } 
}while(a != 100);
 }
 out.println(sum);
\end{lstlisting}
\subsubsection{Последний элемент нужно учитывать}
\begin{lstlisting}[language=Java] 
int sum = 0;
int a;
do {
    a = in.nextInt();
    if (a > 20){
        sum += a;
    }
}while(a != 100);
\end{lstlisting}
\begin{lstlisting}[language=Java] 
int a = 1;
int sum = 0;
while (a != 100){
    a = in.nextInt();
    if (a > 20){
        sum +=a;
    }
} 
\end{lstlisting}
\subsubsection{Еще пример}
\begin{task}
	Дано натуральное число x, найти количество единиц в троичном представлении числа.
\end{task}
\begin{lstlisting}[language=Java] 
int x = in.nextInt();
int k = 0;
while (x != 0){
    if (x % 3 == 1){
        k++;
    }
    x /= 3;
}
\end{lstlisting}
\section*{13.10.22}
\section{Массивы.}
Массивы - совокупность однотипных данных, имеющих общее имя, при этом каждый элемент имеет
уникальный номер, который называется индексом.
Данные в массиве надо обрабатывать с помошью цикла.
\subsection{Описание и создание массива}
\subsubsection{Описание}
\begin{lstlisting}[language=Java] 
DataType [] nameArray; 
\end{lstlisting}
\begin{lstlisting}[language=Java] 
int [] a; 
\end{lstlisting}
\subsubsection{Создание}
\begin{lstlisting}[language=Java] 
nameArray = new DataType[length]; 
\end{lstlisting}
\begin{lstlisting}[language=Java] 
a = new int[10] 
\end{lstlisting}
\subsubsection{Прошлые действия в одном}
\begin{lstlisting}[language=Java] 
int []a = new int[10]; 
\end{lstlisting}
\subsection{Заполнение массива.}
Обнулим массив.
\begin{lstlisting}[language=Java] 
for(int i = 0; i < a.length;i++){
    a[i] = 0;
}
\end{lstlisting} 
Заполним элементами последовательности.(2,4,6,8)
\begin{lstlisting}[language=Java] 
int[0] = 2;
for(int i = 1; i < a.length;i ++){
    a[i] = a[i - 1] + 2;
} 
\end{lstlisting} 
Ввод элементов массива с клавиатуры.
\begin{lstlisting}[language=Java] 
for (int i = 0; i < a.length; i++){
    a[i] = in.nextInt();
}
\end{lstlisting} 
Если в квадратых скобках пишем что-то кроме i, то проверяем не вышли ли из границ массива.
\subsection{Вывод массива на экран.}
\begin{lstlisting}[language=Java] 
for (int i = 0; i < a.length ; i++){
    out.print(a[i] + " ");
} 
out.println();
\end{lstlisting} 
\subsection{Подсчет элементов}
Сумма отрицательных элементов
\begin{lstlisting}[language=Java] 
int sum = 0;
for (int i = 0; i < a.length; i++){
    if (a[i] < 0){
        sum += a[i];
    }
}
out.println(sum);
\end{lstlisting} 
\section{Перестановки}
Пусть есть две ячейки памяти, надо поменять их местами.
\begin{lstlisting}[language=Java] 
int x = 10;
int y = 5;
int z =x;
x = y;
y = z;
\end{lstlisting} 
\section{Поиск минимума и максимума.}
Можно не хранить значение переменной max без условия.
\begin{lstlisting}[language=Java] 
int imax = 0;
for (int i =1 ;i < a.length;i++){
    if (a[i] > a[imax])
        imax = i;
}
\end{lstlisting} 
С условием. Есть тонкости
\begin{enumerate}
    \item Нельзя 0 элемент считать значением максимума
    \item Если известно ограниченние на диапозон, в качестве начального значения берем число,
        выходящее за диапозон.
    \item Если неизвестно, то надо найти первый подходящий элемент.
\end{enumerate}
\section{Заполнение массива случайными числами}
\begin{enumerate}
    \item Подключаем библиотеку Random
        \begin{lstlisting}[language=Java] 
import java.util.Random 
        \end{lstlisting} 
        \item Создать объект класса Random
\begin{lstlisting}[language=Java] 
Random rnd = new Random(0);
\end{lstlisting} 
\item Используем этот объект
\begin{lstlisting}[language=Java] 
rnd.nextInt(k); 
\end{lstlisting} 
Выдает случайное целове число от 0 до k-1.
\end{enumerate}
\section{20.10.20}
\subsection{Однопроходные алгоритмы.}
Перед тем, как сохранить входные (и текущие) данные в массив, нужно ответить на вопрос "нужно ли проходить по данным больше двух раз". Если нужно, то храниим, иначе не храним.
\subsection{Обработка строк}
Строка(String) - некоторое количество символов, стоящее в ряд.
\begin{lstlisting}[language=Java] 
String s = "jopa";
\end{lstlisting} 
Строки неизменяемый тип данных.
\begin{lstlisting}[language=Java] 
s = s + " piska";
\end{lstlisting} 
Создается новая строка.
\begin{lstlisting}[language=Java] 
s.length();
\end{lstlisting} 
Длина строки.
\begin{lstlisting}[language=Java] 
String h = "";
\end{lstlisting} 
Пустая строка.
\begin{lstlisting}[language=Java] 
s.substring(begin) 
\end{lstlisting} 
Подстрока начиная с позиции begin. Индексы элементов строки начинаются с нуля.
\begin{lstlisting}[language=Java] 
s.substring(begin,end);
\end{lstlisting} 
Часть строки с begin до end (не включая).
\begin{lstlisting}[language=Java] 
s.indexOf(chto);
\end{lstlisting} 
Номер позиции первого вхождения chto в строку s. Результат равен -1, если такого нет.
\begin{lstlisting}[language=Java] 
s.charAt(n) 
\end{lstlisting} 
Символ строки с указанным номером.\\
Строки нельзя сравниваь операциями сравнения.
\begin{lstlisting}[language=Java] 
s.equals(s1);
\end{lstlisting} 
Равенство строк.
\begin{lstlisting}[language=Java] 
s.compareTo(s1);
\end{lstlisting} 
Эта штука сравнивает в лексикографическом порядке.
\begin{lstlisting}[language=Java] 
in.nexLine();
\end{lstlisting} 
Ввод строки с клавиатуры.
\subsubsection{Пример}
Дана строка и число. Хотим счиать с клавиатуры.
\begin{lstlisting}[language=Java] 
int x = in.nextInt();
String s = in.nextLine();
\end{lstlisting} 
Не будет работать.
\begin{lstlisting}[language=Java] 
int x = in.nextInt();
in.nexLine();
String s = in.nextLine();
\end{lstlisting} 
Будет работать.
\section{27.10.22}
\subsection{Подпрограммы(статические методы)}
\subsection{Подпрограммы без параметров}
\begin{lstlisting}[language=Java] 
public static resultType name() {
    body;
    return result;
}
\end{lstlisting} 
void - подпрограмма ничего не возвращает. Хороший стиль программирования - ровно один return в последней строке.
\subsubsection{Вызов подпрограммы.}
\begin{lstlisting}[language=Java] 
name();
\end{lstlisting} 
\subsubsection{Примеры}
\begin{task}
    Вводим двухзначное число, возвращаем $\Sigma$ цифр.
\end{task}
\begin{lstlisting}[language=Java] 
public static int sumDigits(){
    int x = in.nextInt();
    return x/10 + x%10;
}
\end{lstlisting} 
\begin{task}
    Вводим целое число, проверяем на четность.
\end{task}
\begin{lstlisting}[language=Java] 
public static isOdd() {
    int x = in.nextIn();
    return x%2 != 0;
}
\end{lstlisting} 
\begin{task}
    Ввести имя и вывести приветствие.
\end{task}
\begin{lstlisting}[language=Java] 
public static void sayHello() {
    String name = in.nextLine();
    out.println("Hello, "+name+"!!!!!!!!!");
}
\end{lstlisting} 
\subsection{Подпрограммы с параметрами.}
\begin{lstlisting}[language=Java] 
public static typeResult name(type name,type name){
    body;
} 
\end{lstlisting} 
\subsubsection{Пример}
\begin{task}
    Найти максимум двух целых чисел.
\end{task}
\begin{lstlisting}[language=Java] 
static int max(int a,int b) {
    int result;
    if (a > b){
        max = a;
    }
    else{
        max = b;
    }
    return result;
}
\end{lstlisting} 
Параметры передаются по значению.
\begin{lstlisting}[language=Java] 
f(x) 
\end{lstlisting}
x не меняется точно.
\begin{task}
    Подпрограмма заполнения массива четными числами от двух.
\end{task}
\begin{lstlisting}[language=Java] 
static void fillArray(int [] a) {
    a[0] = 2;
    for (int i = 1;i < a.length();i++){
       a[i] = a[i - 1] + 2;
    }
}
\end{lstlisting} 
\section{03.11.2022}
\subsection{Рекурсия.}
Рекурсия - это обращение к самому себе.
\begin{task}
    Факториал.
\end{task}
\begin{lstlisting}[language=Java] 
static int fac(int n) {
    if (n == 1) {
        return 1;
    }
    return n * fac(n - 1);
}
\end{lstlisting} 
\subsection{Структура реуррентной программы}
\begin{enumerate}
    \item Провекрка условия продолжения рекурсии.
    \item  Если условие выполнилост
        \begin{enumerate}
            \item Действие
            \item Вызов самой себя.
        \end{enumerate}
\end{enumerate}
\begin{task}
    С клавиатуры вводится последовательность целых чисел, которая кончается нулем.
    Посчитать количество четных чисел.
\end{task}
\begin{lstlisting}[language=Java] 
static int count() {
    int x = in.nextInt();
    if (x  == 0) {
        return 0;
    }

    else {
        if (x % 2 == 0) {
            return 1 + count();
        }
        return count();
    }
}
\end{lstlisting} 
\begin{task}
    Вывести на экран вертикально данно числа десятичные числа.
\end{task}
\begin{lstlisting}[language=Java] 
static void digit(int n) {
    if (n > 0) {
        digit(n / 10);
        out.println(n % 10);
    }
}
\end{lstlisting} 
\section{10.11.2022}
\subsection{Объектно-ориентированное программирование.}
Три кита (парадигмы) ОПП:
\begin{itemize}
    \item Абстрация
    \item Инкапсуляция
    \item Наследование
    \item Полиморфизм
\end{itemize}
\end{document}
