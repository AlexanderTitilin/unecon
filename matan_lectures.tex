\documentclass{article}
\usepackage[utf8]{inputenc}
\usepackage[T2A]{fontenc}
\usepackage[russian]{babel}
\usepackage{amsthm}
\usepackage{mathtools}
\usepackage{hyperref}
\usepackage{tikz}
\newtheorem{theorem}{Теорема}
\newtheorem{corollary}{Следствие}[theorem]
\newtheorem{lemma}[theorem]{Лемма}
\hypersetup{
    colorlinks=true,
    linkcolor=blue,
    filecolor=magenta,      
    urlcolor=cyan,
    pdftitle={Matan Lectures},
    pdfpagemode=FullScreen,
}
\title{Лекции по математическому анализу.}
\author{Александр Титилин}
\date{}
\begin{document}
\maketitle
\tableofcontents
% 	\[
% 		a \ge 0 , \exists \alpha \ge 0 : \alpha^2 = a
% 		.\]
% \end{theorem}
% \begin{proof}
% 	Пусть $A = \{x \mid x \ge  0 , x^2 \le a\}, B = \{x \mid x \ge  0, x^2 \ge  a\}$.
%
% 	A левее B ($\forall  \alpha \in A \forall  \beta \in B, \alpha \le \beta $) Доказываем от противного
%     Пусть $\exists \alpha \in A \exists  \beta \in B: \alpha > \beta \implies 
%     \alpha^2 > \beta^2$ Противоречие.
%
%     $A \neq \emptyset$ так как  $0 \in A$
%
%      $B \neq \emptyset$ Так как $a + 1 \in B$
%
%      По аксиоме полноты  $\forall a \in A, \forall  b \in B : a \le c \le b$.
%      $c^2 = a$ От противного. 
%      \begin{enumerate}
%          \item Пусть $c^2 < a$ Доказать, что $\exists n \in \mathbb{N}$ $(c + \frac{1}{n })^2 < a$
%              Противоречие $c + \frac{1}{n} \in A$ 
%         \item
%             Пусть $c^2 > a$ Доказать, что $\exists  n \in \mathbb{N}: (c - \frac{1}{n} > a) \implies
%             c - \frac{1}{b} \in B$ 
%              $$
%      \end{enumerate}
% \end{proof}
\section{Предел последовательности.}
\subsection{Окрестность точки.}
Окрестность точки a - это произвольный открытый промежуток, содержащий точку a.
\subsection{окрестность}
$U_{\epsilon}(a) = ( e - a  ; e + a )$
\subsection{Определение предела. Геометрическое}
Число а называют пределом последовательности $(x_n)$ если в любой окрестности точки а,
содержатся все члены  $x_n$, начиная с некоторого.
\subsection{Определение предела. Еще одно}
а является пределом $x_n$, если в любой симметричной последовательности точки а содержатся все
члены последовательности начиная с некоторого.
\subsection{Определение предела, еще одно с кванторами, нормальное}
а является пределом $x_n$, если
\[
	\forall \epsilon > 0 \exists n_0 \forall n \ge n_0:~ \mid x_n - a \mid < \epsilon \Leftrightarrow
	x_n \in (a-\epsilon,a+\epsilon)
	.\]
\subsection{Запись предела}
\[
	\lim_{n \to \infty} x_n = a
	.\]
\subsection{Примеры}
\begin{enumerate}
	\item $a_n = 1$. Предел 1, так как все члены последовательности лежат в окрестности 1.
	\item  $a_n = \frac{1}{n}$ Предел 0. a,b концы окрестности .
	      $\exists  n_0 \forall  n \ge n_0 : x_n \in (a,b)$. $n_0 = $ любое число $> \frac{1}{b}$
	\item
	      $x_n = \frac{n}{2n^2 + 1}$

	      $(a , b ) $ - окрестность 0.

	      \[
		      \exists  n_0  \forall n \ge  n_0
		      .\]
	      Надо доказать, что $x_n < b$

	      \[
		      \frac{n}{n^2 + 1} < b
		      .\]
	\item $x_n = \frac{2n + 1}{3n + 2}$ Предел $\frac{2}{3}$
	      \[
		      \frac{2n + 1}{3n + 2} = \frac{2 + \frac{1}{n}}{3 + \frac{2}{n}}
		      .\]
\end{enumerate}
\subsection{Единственность предела}
\begin{theorem}
	У сходящийся последовательности есть только 1 предел.
	\[
		x_n \rightarrow a  \land x_n \rightarrow b \implies a = b
		.\]
\end{theorem}
\begin{proof}
	Пусть $a < b$. Рассмотрим промежутки  $(-\infty , \frac{a + b}{2})$ и $( \frac{a+b}{2},+\infty )$.
	$a \in (-\infty, \frac{a + b }{2}) \land b \in ( \frac{a+b}{2},+\infty )$
	Так как $x_n \rightarrow a ~ \exists  n_0 \forall n \ge  n_0 x_n \in (-\infty,\frac{a + b}{2}),
		\exists x_1 \forall n \ge  n_1 x_n \in (\frac{a + b}{2}, +\infty) , n_2 = \max(n_0,n_1)$
\end{proof}
\subsection{Ограниченные последовательности}
\[
\exists  M \forall n ~ x_n \le  M
.\] 
$x_n$ Ограничена сверху.
\begin{theorem}
    Всякая сходящаяся последовательность ограничена.
\end{theorem}
\begin{proof}
   $x_n \rightarrow a$. Рассмотрим  окрестность (a - 1,a + 1) ,точки a.
   \[
       \exists  n_0 \forall  n \ge  n_0 ~ x_n \in (a - 1, a  + 1)
   .\] 
       $x_{n_0},x_{n_0 + 1}, \dots $  - ограничена
\end{proof}
\subsection{Предельный переход в неравенстве}
\begin{theorem}
    $(x_n),(y_n)$ - последовательности такие, что
     \[
    \forall n : x_n \le  y_n
    .\] 
    \[
    \lim_{n \to \infty} x_n = a
    .\] 
    \[
    \lim_{n \to \infty} y_n = b
    .\] 
    Тогда $a \le  b$

    Заметим, что неравенство, выволняется с некоторого n. В условии теоремы нельзя оба знака неравенства заменить на строгие.
\end{theorem}
\begin{proof}
    От противного. Пусть наши последовательности, такие что $x_n \le  y_n ~ \forall n, a > b$ 
    Рассмотрим $(-\infty, \frac{a + b}{2}) , (\frac{a + b }{2}, +\infty)$. Первый окресность b,
    второй окрестность точки a. Так как $x_n \rightarrow a$ ,то
     \[
    \exists n_0 , \forall n \ge n_0 x_n \in (\frac{a + b}{2}, +\infty)
    .\] 
    \[
        \exists  n_1 , \forall  n\ge n_1 y_{n} \in (-\infty,\frac{a + b}{2})
    .\]
    \[
        n_2 = \max{n_0,n_1}
    .\] 
    Тогда $n \ge  n_2$
\end{proof}
\subsection{Теорема о сжатой последовательности.} \label{menti}
\begin{theorem}
    $(x_n),(y_n),(z_n)$ - последовательности такие, что  $\forall ~  n x_n \le y_{n} \le z_n$.
    Пусть $x_n \rightarrow a, z_n \rightarrow a$. То  $y_n \rightarrow a$
\end{theorem}
\begin{proof}
   Возьмем произвольную окрестность U точки а. Так как $x_n \to a$, то  $\exists n_0 ~ \forall 
   n > n_0 x_n \in U$. $z_n \to a  ~ \exists  n_1 \forall  n > n_1 z_n \in U. n_2 = \max{(n_0,n_1)} 
   ~ \forall n > n_2 x_n \in U z_n \in U$. Но $x_n \le  y_n \le z_n$. Значит $y_n \in U$.
\end{proof}
\subsection{Арифметические операции над последовательностями}
\subsubsection{Бесконечно малые последовательности}
Последовательность называется бесконечно малой, если ее предел равен 0.
\subsubsection{}
\begin{theorem}
$$(x_n) , a \in \mathbb{R}$$. Рассмотри последовательность $\alpha_n = x_n - a$. Тогда $x_n \to a$
\leftrightarrow (\alpha_n)$ бесконечно малая.
\end{theorem}
\begin{proof}
    \[
    \alpha_n \to 0 \leftrightarrow \forall  \epsilon \exists  n_0 \forall n \ge  n_0 \mid \alpha_n \mid <\epsilon
    .\] 
\end{proof}
\subsubsection{Сумма бесконечно малых последовательностей}
\begin{theorem}
    Сумма бесконечно малых бесконечно малая.
\end{theorem}
\begin{proof}
    \[
    \mid x_n + y_n \mid \le \mid x_n \mid + \mid y_n \mid 
    .\] 
Возьмем $\forall \epsilon > 0$. Рассмотри $\frac{e}{2}$.\\ Так как $x_n \to 0, ~ \exists n_1 \forall n
\ge n_1 , \mid x_n \mid < \frac{\epsilon}{2}~$ \\
$y_n \to 0 , \exists  n_2 \forall  n >- n_2 \mid y_n \mid < \frac{\epsilon}{2}$ 
\[
\mid x_n \mid + \mid y_n \mid < \epsilon
.\] 
\end{proof}
\subsubsection{Произведение бесконечно малой на ограниченную}
\begin{theorem}
    $(x_n)$ - бесконечно малая, $(y_n)$ ограниченная   $\rightarrow (x_{n}y_{n})$ бесконечно малая. 
\end{theorem}
\begin{proof}
    \[
    \exists n_0 ~ \forall n \ge n_0 \mid x_n \mid < \frac{\epsilon}{C}
    .\] 
    \[
    \exists C > 0 \forall n \mid y_n \mid < C
    .\] 
    \[
    \mid x_n y_n \mid = \mid x_n \mid \mid y_n \mid < \epsilon
    .\] 
\end{proof}
\subsubsection{Теорема о пределе суммы последовательности}
\begin{theorem}
Если 

    \[
    
    x_n \to a
    .\] 
    \[
    y_n \to b
    .\] 
    То
    \[
    x_n + y_n \to a + b
    .\] 
\end{theorem}
\begin{proof}
    $\alpha_n = x_n - a$,  $\beta_n = y_n - b$ бесконечно малые. 
    Рассмотрим  сумму этимх последовательностей $(x_n + y_n) - (a + b) = \alpha_n + \beta_n$. Вторая сумма бесконечно малая, следовательно  $x_n + y_n \to a + b$
\end{proof}
\subsubsection{Теорема о пределе произведения последовательностей}
\begin{theorem}
    Если $x_n \to a, y_n \to b $ то $ x_n y_n \to ab$
    \[
    x_n y_n = ab + a\beta_n + b \alpha_n + \alpha_n \beta_n
    .\] 
    Три последних слагаемых бесконечно малые.
\end{theorem}
\subsubsection{Теорема о пределе частного}
 \begin{theorem}
Если $y_n \to b$, $\frac{1}{y_n} - \frac{1}{b} \to 0$
\end{theorem}
\begin{proof}
   \[
       \frac{b - y_n}{y_n- b} = (b - y_n)  \frac{1}{b}\frac{1}{y_n}
   .\] 
   Достаточно доказать, что $\frac{1}{y_n}$ ограничена.
\end{proof}
\begin{theorem}
    Если $x_n \to a, y_n \to b , \forall n ~ y_n \neq 0, b\neq_0$, тогда
    $\frac{x_n}{y_n} \to \frac{a}{b}$
\end{theorem}
\subsubsection{Предел квадратного корня}
\begin{theorem}
    $x_n ~\forall n ~x_n \ge 0  a \in \mathbb{R} x_n \to a$ тогда $\sqrt{x_n} = \sqrt{a}$
\end{theorem}
\section{Подпоследовательность}
\subsection{Определение}
$(x_n)$ - числовая последовательность. Выбираем любую строго возрастающую последовательность натуральных чисел ($n_1 < n_2  < n_3 \dots $). Рассматриваем последовательность с элементами $x_{n_1}, x_{n_2}, \dots x_{n_{k}} \dots$
\subsection{}
\begin{theorem} \label{1}
    Из всякой последовательности можно выбрать монотонную подпоследовательность.
\end{theorem}
\begin{proof}

   Пусть $x_n$ последовательность, у которой нет возрастающей подпоследовательности. Тогда 
   докажем, что нее есть убывающая подпоследовательность. Если нет возрастающей подпоследовательность, то
   есть член, все члены с индексами больше него, строго меньше него. Назовем его $x_{n_1}$.
   Рассмотри такую подпоследовательность $x_{n +1} , x_{n + 2},\dots$, в ней нет возрастающей подпоследовательности (в противном случае она возрастающая). Раз это так, то в ней есть  $x_{n_2}$, Такой что все члены раньше него меньше него. Мы построили убывающую последовательность.
\end{proof}
\subsection{Теорема Вейерштрасса}
\begin{theorem} \label{weir}
    Всякая монотонная ограниченная последовательность имеет предел.
\end{theorem}
\begin{proof}
    $(x_n)$ возрастает. Пусть A - это множество значений последовтельности  $(x_n)$. $A \neq \emptyset$.
    А ограниченно сверху. Пусть $\alpha = \sup{A}$. По свойству супремума  $\forall \epsilon > 0 \exists x_{n_0}$ такой что $\alpha - x_{n_0} < \epsilon$. Тогда  $\forall  n\ge n_0 ~ \alpha -x_{n} < \epsilon \implies \mid x_n -\alpha \mid < \epsilon x_n \to \alpha$
    Для убывающей самим надо.
\end{proof}
\subsection{Принцип выбора}
\begin{theorem}
    Из любой ограниченной последовательность, сходящуюся подпоследовательность.
\end{theorem}
\begin{proof}
    В полслова. Пусть $x_n$ - ограниченная последовательность. По теореме \ref{1} есть монотонная подпоследовательность, по теореме \ref{weir} нужная подпоследовательность имеет предел.

    Другое
\end{proof}
\section{Примеры}
\begin{enumerate}
    \item 
        \[
        x_n = \frac{2n + 5}{3n - 7} = \frac{2 + \frac{5}{n}}{3 - \frac{7}{n}   } \to \frac{2}{3}
        .\] 
    \item   
        \[
        \lim_{n \to \infty} (\sqrt{n + 1} - \sqrt{n}) = 
        \lim_{n \to \infty} \frac{(\sqrt{n + 1 } - \sqrt{n})(\sqrt{n + 1} + \sqrt{n})}{\sqrt{n + 1}  + \sqrt{n} } = \lim_{n \to \infty} \frac{1}{\sqrt{n + 1}  + \sqrt{n} } =  0
        .\]  
    \item
        \[
        \lim_{n \to \infty} (\sqrt{n^2 + n}  - n) = 
        \lim_{n \to \infty} \frac{n}{\sqrt{n^2 + n}  + n} = 
        \lim_{n \to \infty} \frac{1}{\sqrt{1 + \frac{1}{n}}  + 1} = \frac{1}{2}
        .\] 
    \item \[
            x_1 = \sqrt{2} , x_{n + 1} = \sqrt{2 + x_n} 
    .\] 
    Пусть $x_n \to a$. 
    \[
        x_{n + 1} = \sqrt{2 + x_n} \to a 
    .\] 
    \[
    a  = \sqrt{2 + a} 
    .\] 
       \[
       a = 2
       .\]  
       Доказываем существание предела. Последовательность строго возрастает. Ограничена по теореме \ref{weir}.
       \item
           \[
           x_n = \frac{1}{1*2} + \frac{1}{2*3} + \frac{1}{3*4} + \dots + \frac{1}{n(n+1)} = 
           \frac{1}{1} - \frac{1}{2} + \frac{1}{2} - \frac{1}{3} + \dots + \frac{1}{n} - \frac{1}{n_1}=
           1 - \frac{1}{n+ 1}
           .\] 
    \item
        $x_n = q^n$, если $\mid q \mid < 1$, то $x_n \to 0$.
        Нужно доказать, что  $\forall  \epsilon > 0 \exists n_0 \forall  n\ge n_0 \mid x_n \mid < \epsilon$
        \[
        \mid q \mid ^n < \epsilon
        .\] 
    \[
        n > \log_{\mid q \mid}  \epsilon
    .\] 
\item $(x_n)$ последовательнось положительных  чисел.  Пусть  $\frac{x_{n + 1}}{x_n} \to c < 1$. Тогда $x_n \to 0$ \label{t}
\end{enumerate}
    \begin{corollary}
        \[
        \mid q \mid < 1 , q^n \to 0
        .\] 
        \[
        x_n = \mid q \mid ^n
        \frac{x_{n + 1}}{x_n} = \mid q \mid \to \mid q \mid < 1
        .\] 
        \[
        \mid q \mid ^n \to 0
        .\] 
    \end{corollary}
        \begin{corollary}
            \[
            x_n = \frac{a^n}{n!}
            .\] 
            \[
                \frac{x_{n + 1}}{x_n} = \frac{a}{n + 1}
            .\] 
        \end{corollary}
        \begin{corollary}
            $a > 1$
             \[
            x_n = \frac{n^k}{a^n}
            .\] 
            \[
                \frac{x_{n+1}}{x_n} = \frac{(1 + \frac{1}{n})^k}{a} < 1
            .\] 
        \end{corollary}
        \begin{proof}\ref{t}
   \[
   c < 1
   .\]  
   Рассмотри произвольное q, такое что $c < q < 1$. Рассмотрим промежуток  $(-\infty,q)$, это окрстность точки с. Так как отношение стремится к c, то  $\exists  n_0 \forall  n\ge n \frac{x_{n + 1}}{x_n}
   \in (-\infty,q)$ 
   \[
       \frac{x_{n_0 + 1}}{x_{n_0}} < q
   .\] 
   \[
       \frac{x_{n_0 + 2}}{x_{n_0 + 1}} < q
   .\] 
   \[
   \[
       \frac{x_{n_0 + k}}{x_{n_0 + k + 1}} < q
   .\] 
   Перемножили все.
   \[
       \frac{x_{n_0 + k}}{x_{n_0}} < q^k
   .\] 
   \[
      0 <  x_{n_0 + k} < x_{n_0} * q^k
   .\] 
   По \ref{menti}
   \[
       x_{n_0 + k} \to 0
   .\] 
\end{proof}
\section{Неравенство Бернулли по индукции} \label{ber}
   \[
       (1 + a)^n (1 + a) \ge (1 +na)(1 + a)
   .\] 
   \[
   (1 + a)^{n + 1} \ge  1 + a + na + na^2
   .\] 
   \[
   1 + a + na + na^2 > 1 + a + na
   .\] 
\section{}
\begin{theorem}
    $x_n  = (1 + \frac{1}{n})^n$ имеет предел.
\end{theorem}
\begin{proof}
            Докажем, что $( x_n )$ возрастает.
            \[
                \frac{x_{n + 1}}{x_n} = \frac{(n + 2)^{n + 1}}{(n + 1)^{n + 1}} \frac{n^n}{(n + 1)^{n}}=
                \frac{(n^2 + 2n)^{n + 1}}{( (n + 1)^2 )^{n + 1}} * \frac{n + 1}{n}
            .\] 
            \[
                (\frac{n^2 + 2n}{n^2 + 2n + 1})^{n + 1} * \frac{n + 1}{n} =
                (1 - \frac{1}{n^2 + 2n + 1}) * \frac{n+1}{n} > 1 - (n + 1) \frac{1}{n^2+2n +1} * \frac{n+1}{n}
            .\] 
            \[
            = ( 1 - \frac{1}{n + 1}) * \frac{n + 1}{n} = \frac{n}{n+1} * \frac{n + 1}{n} = 1
            .\] 
            Докажем, что последовательность ограниченна сверху.
            \[
                x_n = (1 + \frac{1}{n})^n = \sum_{k=0}^{n} C_{n}^k *1 + \frac{1}{n}^k =
                \sum_{k = 0}^{n} \frac{n (n - 1) (n -2) \dots (n - (n - k))}{n^k} * \frac{1}{k!}
            .\] 
            \[
                \sum_{k=0}^{n} (1 - \frac{1}{n}) (1 - \frac{2}{n}) \dots * (1-  \frac{k - 1}{n})*\frac{1}{k!} < \sum_{k = 1}^{n} \frac{1}{k}
            .\] 
            \[
            = 1 + 1 + \frac{1}{2!} + \frac{1}{3!} \dots \frac{1}{n!} <
            .\] 
            \[
                < 1 + (1 + \frac{1}{2} + (\frac{1}{2})^2 + \dots + (\frac{1}{n})^{n -1}) < 3
            .\] 
            По \ref{weir} последовательность имеет предел, назовем его $e$.
\end{proof}
\section{} \label{h1}
\[
x_n \to b
.\] 
\[
a > 0, a \neq 1
.\] 
\[
    a^{x_n} \to a^b
.\] 
\section{} \label{h2}
\[
x_n > 0
.\] 
\[
x_n \to b > 0
.\] 
\[
    \log_a{x_n} \to \log_a{b}
.\] 
\section{}
\[
\alpha > 0
.\] 
\[
    x_n \to b \implies x_{n}^{\alpha} \to b^{\alpha}
.\] 
\section{}
\begin{theorem} \label{sin}
    \[
        \mid \sin{x} \mid = \mid x \mid
    .\] 
\end{theorem}
\begin{proof}
    При $0 < x < \frac{\pi}{2}$
    Картинку потом нарисую.
\end{proof}
\section{}
\begin{theorem}
    \[
        x_n \to b \implies \sin{x_n} \to \sin{a}
    .\] 
\end{theorem}
\begin{proof}
    \[
        \sin{x_n} - \sin{a} \to 0
    .\] 
    \[
        \mid \sin{x_n} - \sin{a} \mid = \mid \sin{\frac{x_n - a}{2}} \cos{\frac{x_n+a}{2}} \mid \le 
        2 \mid \frac{x_n - a}{2} \mid = |x_n - a \mid
    .\] 
    \[
        0 \le \mid \sin{x_n} - \sin{a} \mid \le  \mid x_n -a \mid
    .\] 
    По \ref{menti} $\sin{x_n} \to \sin{a}$
\end{proof}
\section{Задачка}
\[
\lim_{n \to \infty} (\frac{n + 3}{n + 1})^n =
\lim_{n \to \infty} (1 - \frac{2}{n + 1})^n = 
\lim_{n \to \infty} ((1 + \frac{2}{n + 1})^{\frac{n + 1}{2}})^{\frac{2n}{n + 1}}
.\] 
Тупо 1 добавили и вычли.
\section{}
\[
    (1 + \frac{1}{n})^n \to e
.\] 
\[
x_n = \frac{1}{n} \to 0
.\] 
\[
    (1 + x_n)^{\frac{1}{x_n}} \to e
.\] 
\section{}
\[
    \lim_{n \to \infty} \frac{\log_{a}{n}}{n^k} = 0
.\] 
\section{Бесконечно большие последовательности}
Попробуем дать точный смысл записи $x_n \to + \infty, x_n \to -\infty,x_n \to \infty$.
\subsection{}
\[
x_n \to +\infty \iff \e \forall C ~ \exists n_0 ~ \forall n \ge  n_0 x_n > C
.\] 
\subsection{}
\[
x_n \to -\infty \iff \forall C ~ \exists  ~ n_0 ~ \forall n > n_0 ~ x_n < C
.\] 
\subsection{}
\[
x_n \to \infty \iff \mid x_n \mid \to + \infty
.\] 
\subsection{}
\begin{theorem}
    Пусть $(x_n)$ такова, что  $\forall x_n \neq 0$, тогда $(x_n)$ - бесконечно большая
     $\iff \frac{1}{x_n}$ бесконечно малая.
\end{theorem}
\begin{proof}
\[
x_n \to +\infty \iff \e \forall C ~ \exists n_0 ~ \forall n \ge  n_0 \mid x_n \mid > C
.\] 
\[
\frac{1}{x_n} \to 0  \iff \forall  \epsilon \exists  n_0 ~ \forall n \ge  n_0 \frac{1}{\mid x_n \mid} <\epsilon
.\] 
\[
\mid x_n \mid > \frac{1}{\epsilon}
.\] 
\end{proof}
\section{Расширеннная прямая}
$\overline{\mathbb{R}}$ - это $\mathbb{R} \cup \{+\infty,-\infty\}$
\[
-\infty < \infty
.\] 
\[
a \in \mathbb{R} , ~ -\infty < a < +\infty
.\] 
\section{Предел функций.}
\subsection{Предельная точка}
$X \subset \mathbb{R}, a \in X$. Точка а называется предельной точкой множества X, если в любой окрестности точки а, есть хотя бы одно число из X, отличное от а.
\subsection{}
\begin{theorem}
    a - предельная точка множества D $\iff$ Если существует  $(x_n)$ точек множества D
    отличных от а, такая что $x_n \to a$
\end{theorem}
\begin{proof}
    \begin{enumerate}
        \item Пусть а предельная точка D.
            Смотрим промежуток $(a-1;a+1)$. Рассмотрим  $x_1 \in (a-1;a+1),x_1 \neq a, x_1 \in D$. $
            x_2 \in (a - \frac{1}{2},a + \frac{1}{2}), x_2 \neq a , x_2 \in D$. И так далее, мы построили последовательность такую, что $\forall x_n \in D,x x_n \neq a, a - \frac{1}{n} < x_n < a + \frac{1}{n}$. По \ref{menti} $x_n \to a$
            \item
                Пусть  $(x_n)$ такова, что  $x_n \in D,x_n \neq a, x_n \to a$. Взяли произвольну окрестностность а $\exists n_0  ~ x_{n_0} \in U(a)$
    \end{enumerate}
\end{proof}
\subsubsection{Пример}
Возьмем $D = [0;1)$. Найдем все его предельные точки. Это все точки из  $[0;1)$
\subsubsection{Пример}
Возьмем за  $D = [0;1) \cup \{2\}$.
\subsection{Определение предела функции}
Пусть $f : D \to \mathbb{R}$, а - предельная точка множества D. Число A называется пределом функции f в точке a,
если $\forall (x_n)$
\[
\begin{cases}
    \forall x_n \neq a\\
    \forall  x_n \in D\\
    x_n \to a\\
\end{cases}
\implies f(x_n) \to A
.\] 
\end{document} 
