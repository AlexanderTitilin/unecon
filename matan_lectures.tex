\documentclass[a4paper]{article}
\usepackage[utf8]{inputenc}
\usepackage[T2A]{fontenc}
\usepackage[russian]{babel}
\usepackage{amsthm}
\usepackage{mathtools}
\usepackage{amssymb}
\newtheorem{theorem}{Теорема}
\newtheorem{definition}{Определение}
\newtheorem{corollary}{Следствие}[theorem]
\newtheorem{lemma}[theorem]{Лемма}
\title{Лекции по математическому анализу.}
\author{Александр Титилин}
\date{}
\begin{document}
\maketitle
\tableofcontents
% 	\[
% 		a \ge 0 , \exists \alpha \ge 0 : \alpha^2 = a
% 		.\]
% \end{theorem}
% \begin{proof}
% 	Пусть $A = \{x \mid x \ge  0 , x^2 \le a\}, B = \{x \mid x \ge  0, x^2 \ge  a\}$.
%
% 	A левее B ($\forall  \alpha \in A \forall  \beta \in B, \alpha \le \beta $) Доказываем от противного
%     Пусть $\exists \alpha \in A \exists  \beta \in B: \alpha > \beta \implies 
%     \alpha^2 > \beta^2$ Противоречие.
%
%     $A \neq \emptyset$ так как  $0 \in A$
%
%      $B \neq \emptyset$ Так как $a + 1 \in B$
%
%      По аксиоме полноты  $\forall a \in A, \forall  b \in B : a \le c \le b$.
%      $c^2 = a$ От противного. 
%      \begin{enumerate}
%          \item Пусть $c^2 < a$ Доказать, что $\exists n \in \mathbb{N}$ $(c + \frac{1}{n })^2 < a$ Противоречие $c + \frac{1}{n} \in A$ 
%         \item
%             Пусть $c^2 > a$ Доказать, что $\exists  n \in \mathbb{N}: (c - \frac{1}{n} > a) \implies
%             c - \frac{1}{b} \in B$ 
%              $$
%      \end{enumerate}
% \end{proof}
\section{Предел последовательности.}
\subsection{Окрестность точки.}
Окрестность точки a - это произвольный открытый промежуток, содержащий точку a.
\subsection{окрестность}
$U_{\epsilon}(a) = ( e - a  ; e + a )$
\subsection{Определение предела. Геометрическое}
Число а называют пределом последовательности $(x_n)$ если в любой окрестности точки а,
содержатся все члены  $x_n$, начиная с некоторого.
\subsection{Определение предела. Еще одно}
а является пределом $x_n$, если в любой симметричной последовательности точки а содержатся все
члены последовательности начиная с некоторого.
\subsection{Определение предела, еще одно с кванторами, нормальное}
а является пределом $x_n$, если
\[
	\forall \epsilon > 0 \exists n_0 \forall n \ge n_0:~ \mid x_n - a \mid < \epsilon \Leftrightarrow
	x_n \in (a-\epsilon,a+\epsilon)
	.\]
\subsection{Запись предела}
\[
	\lim_{n \to \infty} x_n = a
	.\]
\subsection{Примеры}
\begin{enumerate}
	\item $a_n = 1$. Предел 1, так как все члены последовательности лежат в окрестности 1.
	\item  $a_n = \frac{1}{n}$ Предел 0. a,b концы окрестности .
	      $\exists  n_0 \forall  n \ge n_0 : x_n \in (a,b)$. $n_0 = $ любое число $> \frac{1}{b}$
	\item
	      $x_n = \frac{n}{2n^2 + 1}$

	      $(a , b ) $ - окрестность 0.

	      \[
		      \exists  n_0  \forall n \ge  n_0
		      .\]
	      Надо доказать, что $x_n < b$

	      \[
		      \frac{n}{n^2 + 1} < b
		      .\]
	\item $x_n = \frac{2n + 1}{3n + 2}$ Предел $\frac{2}{3}$
	      \[
		      \frac{2n + 1}{3n + 2} = \frac{2 + \frac{1}{n}}{3 + \frac{2}{n}}
		      .\]
\end{enumerate}
\subsection{Единственность предела}
\begin{theorem}
	У сходящийся последовательности есть только 1 предел.
	\[
		x_n \rightarrow a  \land x_n \rightarrow b \implies a = b
		.\]
\end{theorem}
\begin{proof}
	Пусть $a < b$. Рассмотрим промежутки  $(-\infty , \frac{a + b}{2})$ и $( \frac{a+b}{2},+\infty )$.
	$a \in (-\infty, \frac{a + b }{2}) \land b \in ( \frac{a+b}{2},+\infty )$
	Так как $x_n \rightarrow a ~ \exists  n_0 \forall n \ge  n_0 x_n \in (-\infty,\frac{a + b}{2}),
		\exists x_1 \forall n \ge  n_1 x_n \in (\frac{a + b}{2}, +\infty) , n_2 = \max(n_0,n_1)$
\end{proof}
\subsection{Ограниченные последовательности}
\[
	\exists  M \forall n ~ x_n \le  M
	.\]
$x_n$ Ограничена сверху.
\begin{theorem}
	Всякая сходящаяся последовательность ограничена.
\end{theorem}
\begin{proof}
	$x_n \rightarrow a$. Рассмотрим  окрестность (a - 1,a + 1) ,точки a.
	\[
		\exists  n_0 \forall  n \ge  n_0 ~ x_n \in (a - 1, a  + 1)
		.\]
	$x_{n_0},x_{n_0 + 1}, \dots $  - ограничена
\end{proof}
\subsection{Предельный переход в неравенстве}
\begin{theorem}
	$(x_n),(y_n)$ - последовательности такие, что
	\[
		\forall n : x_n \le  y_n
		.\]
	\[
		\lim_{n \to \infty} x_n = a
		.\]
	\[
		\lim_{n \to \infty} y_n = b
		.\]
	Тогда $a \le  b$

	Заметим, что неравенство, выволняется с некоторого n. В условии теоремы нельзя оба знака неравенства заменить на строгие.
\end{theorem}
\begin{proof}
	От противного. Пусть наши последовательности, такие что $x_n \le  y_n ~ \forall n, a > b$
	Рассмотрим $(-\infty, \frac{a + b}{2}) , (\frac{a + b }{2}, +\infty)$. Первый окресность b,
	второй окрестность точки a. Так как $x_n \rightarrow a$ ,то
	\[
		\exists n_0 , \forall n \ge n_0 x_n \in (\frac{a + b}{2}, +\infty)
		.\]
	\[
		\exists  n_1 , \forall  n\ge n_1 y_{n} \in (-\infty,\frac{a + b}{2})
		.\]
	\[
		n_2 = \max{n_0,n_1}
		.\]
	Тогда $n \ge  n_2$
\end{proof}
\subsection{Теорема о сжатой последовательности.} \label{menti}
\begin{theorem}
	$(x_n),(y_n),(z_n)$ - последовательности такие, что  $\forall ~  n x_n \le y_{n} \le z_n$.
	Пусть $x_n \rightarrow a, z_n \rightarrow a$. То  $y_n \rightarrow a$
\end{theorem}
\begin{proof}
	Возьмем произвольную окрестность U точки а. Так как $x_n \to a$, то  $\exists n_0 ~ \forall
		n > n_0 x_n \in U$. $z_n \to a  ~ \exists  n_1 \forall  n > n_1 z_n \in U. n_2 = \max{(n_0,n_1)}
		~ \forall n > n_2 x_n \in U z_n \in U$. Но $x_n \le  y_n \le z_n$. Значит $y_n \in U$.
\end{proof}
\subsection{Арифметические операции над последовательностями}
\subsubsection{Бесконечно малые последовательности}
Последовательность называется бесконечно малой, если ее предел равен 0.
\subsubsection{}
\begin{theorem}
	$$(x_n) , a \in \mathbb{R}$$. Рассмотри последовательность $\alpha_n = x_n - a$. Тогда $x_n \to a$
	$\leftrightarrow (\alpha_n)$ бесконечно малая.
\end{theorem}
\begin{proof}
	\[
		\alpha_n \to 0 \leftrightarrow \forall  \epsilon \exists  n_0 \forall n \ge  n_0 \mid \alpha_n \mid <\epsilon
		.\]
\end{proof}
\subsubsection{Сумма бесконечно малых последовательностей}
\begin{theorem}
	Сумма бесконечно малых бесконечно малая.
\end{theorem}
\begin{proof}
	\[
		\mid x_n + y_n \mid \le \mid x_n \mid + \mid y_n \mid
		.\]
	Возьмем $\forall \epsilon > 0$. Рассмотри $\frac{e}{2}$.\\ Так как $x_n \to 0, ~ \exists n_1 \forall n
		\ge n_1 , \mid x_n \mid < \frac{\epsilon}{2}~$ \\
	$y_n \to 0 , \exists  n_2 \forall  n >- n_2 \mid y_n \mid < \frac{\epsilon}{2}$
	\[
		\mid x_n \mid + \mid y_n \mid < \epsilon
		.\]
\end{proof}
\subsubsection{Произведение бесконечно малой на ограниченную}
\begin{theorem}
	$(x_n)$ - бесконечно малая, $(y_n)$ ограниченная   $\rightarrow (x_{n}y_{n})$ бесконечно малая.
\end{theorem}
\begin{proof}
	\[
		\exists n_0 ~ \forall n \ge n_0 \mid x_n \mid < \frac{\epsilon}{C}
		.\]
	\[
		\exists C > 0 \forall n \mid y_n \mid < C
		.\]
	\[
		\mid x_n y_n \mid = \mid x_n \mid \mid y_n \mid < \epsilon
		.\]
\end{proof}
\subsubsection{Теорема о пределе суммы последовательности}
\begin{theorem}
	Если

	\[

		x_n \to a
		.\]
	\[
		y_n \to b
		.\]
	То
	\[
		x_n + y_n \to a + b
		.\]
\end{theorem}
\begin{proof}
	$\alpha_n = x_n - a$,  $\beta_n = y_n - b$ бесконечно малые.
	Рассмотрим  сумму этимх последовательностей $(x_n + y_n) - (a + b) = \alpha_n + \beta_n$. Вторая сумма бесконечно малая, следовательно  $x_n + y_n \to a + b$
\end{proof}
\subsubsection{Теорема о пределе произведения последовательностей}
\begin{theorem}
	Если $x_n \to a, y_n \to b $ то $ x_n y_n \to ab$
	\[
		x_n y_n = ab + a\beta_n + b \alpha_n + \alpha_n \beta_n
		.\]
	Три последних слагаемых бесконечно малые.
\end{theorem}
\subsubsection{Теорема о пределе частного}
\begin{theorem}
	Если $y_n \to b$, $\frac{1}{y_n} - \frac{1}{b} \to 0$
\end{theorem}
\begin{proof}
	\[
		\frac{b - y_n}{y_n- b} = (b - y_n)  \frac{1}{b}\frac{1}{y_n}
		.\]
	Достаточно доказать, что $\frac{1}{y_n}$ ограничена.
\end{proof}
\begin{theorem}
	Если $x_n \to a, y_n \to b , \forall n ~ y_n \neq 0, b\neq_0$, тогда
	$\frac{x_n}{y_n} \to \frac{a}{b}$
\end{theorem}
\subsubsection{Предел квадратного корня}
\begin{theorem}
	$x_n ~\forall n ~x_n \ge 0  a \in \mathbb{R} x_n \to a$ тогда $\sqrt{x_n} = \sqrt{a}$
\end{theorem}
\section{Подпоследовательность}
\subsection{Определение}
$(x_n)$ - числовая последовательность. Выбираем любую строго возрастающую последовательность натуральных чисел ($n_1 < n_2  < n_3 \dots $). Рассматриваем последовательность с элементами $x_{n_1}, x_{n_2}, \dots x_{n_{k}} \dots$
\subsection{}
\begin{theorem} \label{1}
	Из всякой последовательности можно выбрать монотонную подпоследовательность.
\end{theorem}
\begin{proof}

	Пусть $x_n$ последовательность, у которой нет возрастающей подпоследовательности. Тогда
	докажем, что нее есть убывающая подпоследовательность. Если нет возрастающей подпоследовательность, то
	есть член, все члены с индексами больше него, строго меньше него. Назовем его $x_{n_1}$.
	Рассмотри такую подпоследовательность $x_{n +1} , x_{n + 2},\dots$, в ней нет возрастающей подпоследовательности (в противном случае она возрастающая). Раз это так, то в ней есть  $x_{n_2}$, Такой что все члены раньше него меньше него. Мы построили убывающую последовательность.
\end{proof}
\subsection{Теорема Вейерштрасса}
\begin{theorem} \label{weir}
	Всякая монотонная ограниченная последовательность имеет предел.
\end{theorem}
\begin{proof}
	$(x_n)$ возрастает. Пусть A - это множество значений последовтельности  $(x_n)$. $A \neq \emptyset$.
	А ограниченно сверху. Пусть $\alpha = \sup{A}$. По свойству супремума  $\forall \epsilon > 0 \exists x_{n_0}$ такой что $\alpha - x_{n_0} < \epsilon$. Тогда  $\forall  n\ge n_0 ~ \alpha -x_{n} < \epsilon \implies \mid x_n -\alpha \mid < \epsilon x_n \to \alpha$
	Для убывающей самим надо.
\end{proof}
\subsection{Принцип выбора}
\begin{theorem}
	Из любой ограниченной последовательность, сходящуюся подпоследовательность.
\end{theorem}
\begin{proof}
	В полслова. Пусть $x_n$ - ограниченная последовательность. По теореме \ref{1} есть монотонная подпоследовательность, по теореме \ref{weir} нужная подпоследовательность имеет предел.

	Другое
\end{proof}
\section{Примеры}
\begin{enumerate}
	\item
	      \[
		      x_n = \frac{2n + 5}{3n - 7} = \frac{2 + \frac{5}{n}}{3 - \frac{7}{n}   } \to \frac{2}{3}
		      .\]
	\item
	      \[
		      \lim_{n \to \infty} (\sqrt{n + 1} - \sqrt{n}) =
		      \lim_{n \to \infty} \frac{(\sqrt{n + 1 } - \sqrt{n})(\sqrt{n + 1} + \sqrt{n})}{\sqrt{n + 1}  + \sqrt{n} } = \lim_{n \to \infty} \frac{1}{\sqrt{n + 1}  + \sqrt{n} } =  0
		      .\]
	\item
	      \[
		      \lim_{n \to \infty} (\sqrt{n^2 + n}  - n) =
		      \lim_{n \to \infty} \frac{n}{\sqrt{n^2 + n}  + n} =
		      \lim_{n \to \infty} \frac{1}{\sqrt{1 + \frac{1}{n}}  + 1} = \frac{1}{2}
		      .\]
	\item \[
		      x_1 = \sqrt{2} , x_{n + 1} = \sqrt{2 + x_n}
		      .\]
	      Пусть $x_n \to a$.
	      \[
		      x_{n + 1} = \sqrt{2 + x_n} \to a
		      .\]
	      \[
		      a  = \sqrt{2 + a}
		      .\]
	      \[
		      a = 2
		      .\]
	      Доказываем существание предела. Последовательность строго возрастает. Ограничена по теореме \ref{weir}.
	\item
	      \[
		      x_n = \frac{1}{1*2} + \frac{1}{2*3} + \frac{1}{3*4} + \dots + \frac{1}{n(n+1)} =
		      \frac{1}{1} - \frac{1}{2} + \frac{1}{2} - \frac{1}{3} + \dots + \frac{1}{n} - \frac{1}{n_1}=
		      1 - \frac{1}{n+ 1}
		      .\]
	\item
	      $x_n = q^n$, если $\mid q \mid < 1$, то $x_n \to 0$.
	      Нужно доказать, что  $\forall  \epsilon > 0 \exists n_0 \forall  n\ge n_0 \mid x_n \mid < \epsilon$
	      \[
		      \mid q \mid ^n < \epsilon
		      .\]
	      \[
		      n > \log_{\mid q \mid}  \epsilon
		      .\]
	\item $(x_n)$ последовательнось положительных  чисел.  Пусть  $\frac{x_{n + 1}}{x_n} \to c < 1$. Тогда $x_n \to 0$ \label{t}
\end{enumerate}
\begin{corollary}
	\[
		\mid q \mid < 1 , q^n \to 0
		.\]
	\[
		x_n = \mid q \mid ^n
		\frac{x_{n + 1}}{x_n} = \mid q \mid \to \mid q \mid < 1
		.\]
	\[
		\mid q \mid ^n \to 0
		.\]
\end{corollary}
\begin{corollary}
	\[
		x_n = \frac{a^n}{n!}
		.\]
	\[
		\frac{x_{n + 1}}{x_n} = \frac{a}{n + 1}
		.\]
\end{corollary}
\begin{corollary}
	$a > 1$
	\[
		x_n = \frac{n^k}{a^n}
		.\]
	\[
		\frac{x_{n+1}}{x_n} = \frac{(1 + \frac{1}{n})^k}{a} < 1
		.\]
\end{corollary}
\begin{proof}\ref{t}
	\[
		c < 1
		.\]
	Рассмотри произвольное q, такое что $c < q < 1$. Рассмотрим промежуток  $(-\infty,q)$, это окрстность точки с. Так как отношение стремится к c, то  $\exists  n_0 \forall  n\ge n \frac{x_{n + 1}}{x_n}
		\in (-\infty,q)$
	\[
		\frac{x_{n_0 + 1}}{x_{n_0}} < q
		.\]
	\[
		\frac{x_{n_0 + 2}}{x_{n_0 + 1}} < q
		.\]
	\[
		\frac{x_{n_0 + k}}{x_{n_0 + k + 1}} < q
		.\]
	Перемножили все.
	\[
		\frac{x_{n_0 + k}}{x_{n_0}} < q^k
		.\]
	\[
		0 <  x_{n_0 + k} < x_{n_0} * q^k
		.\]
	По \ref{menti}
	\[
		x_{n_0 + k} \to 0
		.\]
\end{proof}
\section{Неравенство Бернулли по индукции} \label{ber}
\[
	(1 + a)^n (1 + a) \ge (1 +na)(1 + a)
	.\]
\[
	(1 + a)^{n + 1} \ge  1 + a + na + na^2
	.\]
\[
	1 + a + na + na^2 > 1 + a + na
	.\]
\section{}
\begin{theorem}
	$x_n  = (1 + \frac{1}{n})^n$ имеет предел.
\end{theorem}
\begin{proof}
	Докажем, что $( x_n )$ возрастает.
	\[
		\frac{x_{n + 1}}{x_n} = \frac{(n + 2)^{n + 1}}{(n + 1)^{n + 1}} \frac{n^n}{(n + 1)^{n}}=
		\frac{(n^2 + 2n)^{n + 1}}{( (n + 1)^2 )^{n + 1}} * \frac{n + 1}{n}
		.\]
	\[
		(\frac{n^2 + 2n}{n^2 + 2n + 1})^{n + 1} * \frac{n + 1}{n} =
		(1 - \frac{1}{n^2 + 2n + 1}) * \frac{n+1}{n} > 1 - (n + 1) \frac{1}{n^2+2n +1} * \frac{n+1}{n}
		.\]
	\[
		= ( 1 - \frac{1}{n + 1}) * \frac{n + 1}{n} = \frac{n}{n+1} * \frac{n + 1}{n} = 1
		.\]
	Докажем, что последовательность ограниченна сверху.
	\[
		x_n = (1 + \frac{1}{n})^n = \sum_{k=0}^{n} C_{n}^k *1 + \frac{1}{n}^k =
		\sum_{k = 0}^{n} \frac{n (n - 1) (n -2) \dots (n - (n - k))}{n^k} * \frac{1}{k!}
		.\]
	\[
		\sum_{k=0}^{n} (1 - \frac{1}{n}) (1 - \frac{2}{n}) \dots * (1-  \frac{k - 1}{n})*\frac{1}{k!} < \sum_{k = 1}^{n} \frac{1}{k}
		.\]
	\[
		= 1 + 1 + \frac{1}{2!} + \frac{1}{3!} \dots \frac{1}{n!} <
		.\]
	\[
		< 1 + (1 + \frac{1}{2} + (\frac{1}{2})^2 + \dots + (\frac{1}{n})^{n -1}) < 3
		.\]
	По \ref{weir} последовательность имеет предел, назовем его $e$.
\end{proof}
\section{} \label{h1}
\[
	x_n \to b
	.\]
\[
	a > 0, a \neq 1
	.\]
\[
	a^{x_n} \to a^b
	.\]
\section{} \label{h2}
\[
	x_n > 0
	.\]
\[
	x_n \to b > 0
	.\]
\[
	\log_a{x_n} \to \log_a{b}
	.\]
\section{}
\[
	\alpha > 0
	.\]
\[
	x_n \to b \implies x_{n}^{\alpha} \to b^{\alpha}
	.\]
\section{}
\begin{theorem} \label{sin}
	\[
		\mid \sin{x} \mid = \mid x \mid
		.\]
\end{theorem}
\begin{proof}
	При $0 < x < \frac{\pi}{2}$
	Картинку потом нарисую.
\end{proof}
\section{}
\begin{theorem}
	\[
		x_n \to b \implies \sin{x_n} \to \sin{a}
		.\]
\end{theorem}
\begin{proof}
	\[
		\sin{x_n} - \sin{a} \to 0
		.\]
	\[
		\mid \sin{x_n} - \sin{a} \mid = \mid \sin{\frac{x_n - a}{2}} \cos{\frac{x_n+a}{2}} \mid \le
		2 \mid \frac{x_n - a}{2} \mid = |x_n - a \mid
		.\]
	\[
		0 \le \mid \sin{x_n} - \sin{a} \mid \le  \mid x_n -a \mid
		.\]
	По \ref{menti} $\sin{x_n} \to \sin{a}$
\end{proof}
\section{Задачка}
\[
	\lim_{n \to \infty} (\frac{n + 3}{n + 1})^n =
	\lim_{n \to \infty} (1 - \frac{2}{n + 1})^n =
	\lim_{n \to \infty} ((1 + \frac{2}{n + 1})^{\frac{n + 1}{2}})^{\frac{2n}{n + 1}}
	.\]
Тупо 1 добавили и вычли.
\section{}
\[
	(1 + \frac{1}{n})^n \to e
	.\]
\[
	x_n = \frac{1}{n} \to 0
	.\]
\[
	(1 + x_n)^{\frac{1}{x_n}} \to e
	.\]
\section{}
\[
	\lim_{n \to \infty} \frac{\log_{a}{n}}{n^k} = 0
	.\]
\section{Бесконечно большие последовательности}
Попробуем дать точный смысл записи $x_n \to + \infty, x_n \to -\infty,x_n \to \infty$.
\subsection{}
\[
	x_n \to +\infty \iff \e \forall C ~ \exists n_0 ~ \forall n \ge  n_0 x_n > C
	.\]
\subsection{}
\[
	x_n \to -\infty \iff \forall C ~ \exists  ~ n_0 ~ \forall n > n_0 ~ x_n < C
	.\]
\subsection{}
\[
	x_n \to \infty \iff \mid x_n \mid \to + \infty
	.\]
\subsection{}
\begin{theorem}
	Пусть $(x_n)$ такова, что  $\forall x_n \neq 0$, тогда $(x_n)$ - бесконечно большая
	$\iff \frac{1}{x_n}$ бесконечно малая.
\end{theorem}
\begin{proof}
	\[
		x_n \to +\infty \iff \e \forall C ~ \exists n_0 ~ \forall n \ge  n_0 \mid x_n \mid > C
		.\]
	\[
		\frac{1}{x_n} \to 0  \iff \forall  \epsilon \exists  n_0 ~ \forall n \ge  n_0 \frac{1}{\mid x_n \mid} <\epsilon
		.\]
	\[
		\mid x_n \mid > \frac{1}{\epsilon}
		.\]
\end{proof}
\section{Расширеннная прямая}
$\overline{\mathbb{R}}$ - это $\mathbb{R} \cup \{+\infty,-\infty\}$
\[
	-\infty < \infty
	.\]
\[
	a \in \mathbb{R} , ~ -\infty < a < +\infty
	.\]
\section{Предел функций.}
\subsection{Предельная точка}
$X \subset \mathbb{R}, a \in X$. Точка а называется предельной точкой множества X, если в любой окрестности точки а, есть хотя бы одно число из X, отличное от а.
\subsection{}
\begin{theorem}
	a - предельная точка множества D $\iff$ Если существует  $(x_n)$ точек множества D
	отличных от а, такая что $x_n \to a$
\end{theorem}
\begin{proof}
	\begin{enumerate}
		\item Пусть а предельная точка D.
		      Смотрим промежуток $(a-1;a+1)$. Рассмотрим  $x_1 \in (a-1;a+1),x_1 \neq a, x_1 \in D$. $
			      x_2 \in (a - \frac{1}{2},a + \frac{1}{2}), x_2 \neq a , x_2 \in D$. И так далее, мы построили последовательность такую, что $\forall x_n \in D,x x_n \neq a, a - \frac{1}{n} < x_n < a + \frac{1}{n}$. По \ref{menti} $x_n \to a$
		\item
		      Пусть  $(x_n)$ такова, что  $x_n \in D,x_n \neq a, x_n \to a$. Взяли произвольну окрестностность а $\exists n_0  ~ x_{n_0} \in U(a)$
	\end{enumerate}
\end{proof}
\subsubsection{Пример}
Возьмем $D = [0;1)$. Найдем все его предельные точки. Это все точки из  $[0;1)$
\subsubsection{Пример}
Возьмем за  $D = [0;1) \cup \{2\}$.
\subsection{Определение предела функции}
Пусть $f : D \to \mathbb{R}$, а - предельная точка множества D. Число A называется пределом функции f в точке a,
если $\forall (x_n)$
\[
	\begin{cases}
		\forall x_n \neq a \\
		\forall  x_n \in D \\
		x_n \to a          \\
	\end{cases}
	\implies f(x_n) \to A
	.\]
\subsection{Запись предела функции}
\[
	\lim_{x \to a} f(x) = A
	.\]
\subsection{}
Пусть $\lim_{x \to a} f(x) = A \land \lim_{x \to a} f(x) = B $
\subsection{}
\begin{theorem}
	$f : D \to \mathbb{R}$,a предельная точка множества D. U - окрестность точки а, тогда предел
	функции в точке существует  $\iff$ существует предел на сужении  $D \cap U$
\end{theorem}
\subsection{Теорема о предельном переходе в неравенстве}
\begin{theorem}
	$f,g : D \to \mathbb{R}$, а предельная точка множества D.
	\[
		\forall  x \in D f(x) \le  f(x)
		.\]
	\[
		\exists \lim_{x \to a} f(x), \lim_{n \to a} g(x)
		.\]
	Тогда $\lim_{x \to a} f(x) \le  \lim_{x \to a} g(x)$
\end{theorem}
\subsection{}\label{menti2}
Теорема \ref{menti}, но про пределы функции.
\subsection{Теорема о пределе суммы, произведения и частного}
Пусть $f,g: D \to \mathbb{R}$,а предельна точка множества D.
Пусть $\lim_{x \to a} f(x) = A, \lim_{x \to a} g(x) = b $
Тогда
\begin{enumerate}
	\item $\lim_{x \to a} f(x) + g(x) = A + B$
	\item $\lim_{x \to a} f(x) * g(x) = A*B$
	\item $\lim_{x \to a} \frac{f(x)}{g(x)} = \frac{A}{B}, g(x) \neq 0, B \neq 0$
\end{enumerate}
\section{Композиция функций для вещественных функций}
\[
	f : D \to \mathbb{R}
	.\]
\[
	g : E \to \mathbb{R}
	.\]
\[
	f(D) \subset E
	.\]
\[
	g \circ f = g(f(x))
	.\]
\subsection{Примеры}
\[
	F(x) = \sin^2{x}
	.\]
\[
	f(x) = \sin{x}
	.\]
\[
	g(x) = x^2
	.\]
\[
	g \circ f = F
	.\]
\[
	f \circ g = \sin x^2
	.\]
\section{Предел композиции.}\label{compose}
\begin{theorem}
	$f : D \to \mathbb{R}$, $g : E \to \mathbb{R}$. Пусть $f(D) \subset E$.
	$а$ пределеьная точка множества $D$  и  $\lim_{x \to a} f(x) = b$. Пусть b предельная точка множества $E$ и $\lim_{t \to b} g(t) = c$. Пусть выполняется
	одно из двух условий.
	\begin{enumerate}
		\item $\exists  U$ точки $a$, такая что $\forall  x \neq a \in U \cap D, f(x) \neq b$
		\item $\lim_{t \to b} g(t) = g(b) $.
	\end{enumerate}
	Тогда $\lim_{x \to a} g(f(x)) =  c$
\end{theorem}
\begin{proof}
	Пусть 1 верно, мы хотим доказать, что $\lim_{x \to a} g(f(x)) = c$
	Возьмем $\forall  (x_n)$ такую что $x_n \in D, x_a \neq a x_n \to a$
	\[
		t_n = f(x_n)
		.\]
	\[
		t_n \to b
		.\]
	Нужно проверить, что $g(f(x_n)) \to c$.
	\[
		\dots
		.\]
\end{proof}
\subsection{Пример}
\[
	\lim_{x \to 0} \frac{\sin{3x}}{x}
	.\]
\[
	3x = t
	.\]
\[
	\frac{\sin{x}}{x} = \frac{\sin{t}}{t/3} = 3 \frac{\sin{t}}{t}
	.\]
\section{Одностронние  пределы.}
\begin{theorem}
	D - промежуток, $f : D \to \mathbb{R}$, а предельная точка.
	$\lim_{x \to a} f(x)$ сущесвтует $\iff$  оба односторонних предела существуют и они равны друг другу.
\end{theorem}
\section{Вычисление пределов}
\subsection{}
$\lim_{x \to c} a^x = a^c$
\begin{proof}
	\begin{enumerate}
		\item $a>1$
		      \[
			      a^x - a^c = a^c(a^{x - c} - 1)
			      .\]
		      Докажем, что $a^{x - c} - 1 \to 0, x \to c$
		      \[
			      t = x - c
			      .\]
		      \[
			      \lim_{t \to 0} (a^t - 1) = 0 ?
			      .\]
		      Докажем, что $a^{t_n} - 1 \to 0$
		      \[
			      \forall \epsilon > 0 \exists o \forall n \ge n_0 1 - \epsilon < a^{t_n} < 1 + \epsilon
			      .\]
		      \[
			      \log_a{( 1 - \epsilon )} < t_n < \log_a{( 1 + \epsilon )}
			      .\]
		      Расссмотрим промежуток $(\log_a{( 1 - \epsilon )};\log_a{( 1 + \epsilon )})$
	\end{enumerate}
\end{proof}
\begin{theorem}
	$\lim_{x \to c} \log_a{x} = \log_a{c}$
\end{theorem}
\begin{theorem}
	$\lim_{x \to c} x^a = c^a$
\end{theorem}
\begin{proof}
	\[
		x^a = e^{\ln{x^a}}= e^{a\ln{x}}
		.\]
	\[
		\lim_{x \to c} e^{a\ln{x}}
		.\]
	\[
		g(x) = e^x
		.\]
	\[
		f(x) = a\ln{x}
		.\]
	По теореме \ref{compose} $\lim_{x \to c} f(x) = a \ln{c}$. $\lim_{x \to a\ln{c}} g(x) = g(a\ln{c}) = e^{a\ln{c}} = c^a$
\end{proof}
\begin{theorem}
	$\lim_{x \to 0} (1 + x)^{\frac{1}{x}}  = e$
\end{theorem}
\begin{proof}
	$\lim_{x \to 0^+} (1 + x)^{\frac{1}{x}}$
	\[
		x_n = \frac{1}{t_n}, t_n \to +\infty
		.\]
	\[
		(1 + \frac{1}{t_n})^{t_n} \to e ??
		.\]
	Докажем, что это справедливо для последовательностей, где $t_n$ состоит из натуральных чисел, по \ref{sd}. Теперь в общем случае.

	$(t_n)$ - любая последовательность  $\to +\infty$. Нужно доказать, что  $(1 + \frac{1}{t_n})^{t_n} \to e$
	\[
		[t_n] \le t_n <  [t_n] +1
		.\]
	\[
		\frac{1}{[t_n] + 1} < \frac{1}{t_n} \le  1\frac{1}{[t_n]}
		.\]
	\[
		1 + \frac{1}{[t_n] + 1} < 1 + \frac{1}{t_n} \le  1 + \frac{1}{[t_n]}
		.\]
	\[
		( 1 + \frac{1}{[t_n] + 1} )^{t_n}< 1  + ( \frac{1}{t_n} )^{t_n} \le  (1 +  \frac{1}{[t_n]} )^{t_n}
		.\]
	Докажем, что левый предел тоже $e$.
	\[
		\forall  (x_n) ~\text{Такой что} ~
		\begin{cases}
			x_n > -1 \\
			x_n < 0  \\
			x_n \to 0
		\end{cases}
		.\]
	Рассмотрим последовательность $t_n = -\frac{1}{x_n} > 0, t_n \to 0$
	\[
		(1 + x_n)^{\frac{1}{x_n}} = (1 - \frac{1}{t_n})^{-t_n} =
		(\frac{t_n}{t_n -1} )^{t_n} = (1 + \frac{1}{t_n -1 })^{t_n - 1} (1 +  \frac{1}{t_n - 1})
		.\]
\end{proof}
\begin{lemma} \label{sd}
	$f : (a,+\infty] \to \mathbb{R}$. Пусть $f(n) \to A$. Тогда для $\forall $ последовательность натуральных чисел $t_n \to + \infty$ выполняется  $f(t_n) \to A$.
\end{lemma}
\begin{proof}
	$\forall  \epsilon > 0 $ неравенство $|f(n) = a| < \epsilon$ для почти всех натуральных чисел. Тогда
	$\forall \epsilon > 0 \mid f(t_n) \to A \mid$

\end{proof}
\begin{theorem}
	$\lim_{x \to 0} \frac{\ln{( 1 + x )}}{x} = 1$
\end{theorem}
\begin{proof}
	\[
		\frac{\ln(1 + x)}{x} = \ln{(1 + x)^{\frac{1}{x}}}
		.\]
\end{proof}
\begin{theorem}
	$\lim_{x \to 0} \frac{a^x - 1}{x} = \ln{a}$
\end{theorem}
\begin{proof}
	\[
		t = a^x + 1
		.\]
	\[
		x = \log_a{( 1 + t )}
		.\]
	\[
		\frac{a^x - 1 }{x} = \frac{t}{\log_a{1 + t}}
		.\]
\end{proof}
\begin{theorem}
	$\lim_{x \to 0} \frac{(1 + x)^{\alpha} - 1}{x}$
\end{theorem}
\begin{proof}
	\[
		(1 + x)^{\alpha} - 1 = t
		.\]
	\[
		\frac{t}{x} = \frac{t}{\ln{( 1 + t )}} * \frac{\ln{(1 + t)}}{x} = \alpha
		.\]
\end{proof}
\begin{theorem}
	$\lim_{x \to a} \sin{x} = \sin{a}$
\end{theorem}
\begin{theorem}
	$\lim_{x \to 0} \frac{\sin{x}}{x} = 1$
\end{theorem}
\begin{proof}
	Пусть $0 < x < \frac{\pi}{2}$
	\[
		\sin{x} < x <  \tg{x}
		.\]
	\[
		1 < \frac{x}{\sin{x}} < \frac{1}{\cos{x}}
		.\]
\end{proof}
\section{Сравнение роста функций}
\subsection{}
\begin{theorem}
	\[
		\lim_{x \to +\infty} \frac{\log_a{x}}{x^\alpha} = 0
		.\]
	\[
		a > 1, \alpha > 0
		.\]
\end{theorem}
\begin{proof}
	Досточно доказать для натурального логорифма, так как $\log_a{x} = \frac{\ln{x}}{\ln{a}}$

	Мы знаем, что $\lim_{n \to \infty} \frac{\log_a{n}}{n^\alpha} = 0$
\end{proof}
\begin{theorem}
	$\mid a \mid < 1, \lim_{x \to + \infty}a^x = 0 $
\end{theorem}
\section{Другое определение предела.}
\begin{definition}
	$f : D \to \mathbb{R}$, a предельная точка множества D.
	A называется пределом f точке a, если
	\[
		\forall \epsilon > 0 | f(x) - A | < \epsilon
		.\]
	Выполняется в некоторой проколотой окрестности
	точки а (т.е открестноть точки а без а).
\end{definition}
\subsection{Примеры употребления термина "вблизи"}
\subsubsection{}
Неравенство $x < 1$ выполняется вблизи 0.
\subsubsection{}
Неравенство $\frac{x^2}{x} < 1$ выполняется вблизи 0.
\section{Непрерывная функция}
f называется непрерывной в точке a.
\begin{enumerate}
	\item Если а предельная точка множества D
	      $\lim_{x \to a} f(x)  = f(a)$
	\item
	      иначе f непрерывна в точке а
\end{enumerate}
$f: D \to \mathbb{R}$ непрерывна в точке а, если
\[
	\forall x_n
	\begin{cases}
		x_n \in D \\
		x_n \to a
	\end{cases}
	.\]
выполняется $f(x_n) \to f(a)$
\[
	\forall  \epsilon > 0 \exists \delta > 0 \forall  x |x - a| <\delta \implies
	|f(x) - f(a)| < \epsilon
	.\]
\[
	|f(x) - A| < \epsilon \iff f(x) \in (A - \epsilon;A+\epsilon)
	.\]
f непрерывна в а, если $\forall $ окрестность V точки $f(a)$
Найдется такая окрестность U точки а, такая что множество значений f на U лежит в V.
\section{Арифметические действия над непрерывными функциями}
\begin{theorem}
	\[
		f,g: D \to \mathbb{R}, a \in D
		.\]
	f и g непрерывны в точке а. Тогда
	\begin{enumerate}
		\item $f + g$ непрерывна в точке а.
		\item  $fg$ непрерывна в точке а.
		\item
		      $\forall  x \in D g(x) \neq 0$ то $\frac{f}{g}$ непрерывны в точке а.
	\end{enumerate}
\end{theorem}
\begin{theorem}
	\[
		f : D \to \mathbb{R}
		.\]
	\[
		g: E \to \mathbb{R}
		.\]
	\[
		f(D) \subset E
		.\]
	Пусть $a \in D$, f непрерывна в точке а, g непрерывна в точке  $f(a)$. Тогда  $f\circ g$ непрерывна в точке а.
\end{theorem}
\begin{proof}
	Если а изолированная точка, то по определению. Если а предельная точка вычисляем предел композиции.
	\[
		\lim_{x \to a} g(f(x)) = g(f(a))
		.\]
	\[
		x_n \to a
		.\]
	f непрерывна в а, $f(x_n) \to f(a)$\\
	g непрерывна в $f(a)$,  $g(f(x_n)) \to f(a)$
\end{proof}
\section{Разрыв первого рода}
Если односторонниие пределы конечные числа и не равны друг другу - функция имеет разрыв первого рода.
\section{Разрыв второго рода}
Если один из односторонних пределов бесконечность или не существует, то функция имеет разрыв второго рода.
\section{}
\begin{theorem}
	Если функция задана на промежутке и непрерывна в каждой точке промежутка, то и обратная функция непрерывна.
\end{theorem}
\section{Теорема Больцано}
\begin{theorem}[О промежуточном значении]
	Пусть функция $f$ задана на промежутке и непрерывна, тогда множеством значений функции является промедуток.
	\[
		\forall  a, b \in D~
		\forall  C
		.\]
	C лежит между $f(a)$ и  $f(b)$ Тогда $\exists  c \in [a,b]$, что $f(c) = C$
\end{theorem}
\begin{proof}
	Достаточно доказать, что $\forall a,b \in D$ таких, что $a < b$ и  $f(a) \neq f(b)$ $\forall  C$ строго между $f(a) , f(b)$.  $\exists c$, таких , что $a < c < b$ и  $f(c)  = C$.
	Начнем с частного случая.
	\begin{enumerate}
		\item Пусть $f(a) < 0, f(b) > 0$, тогда  $\exists  c \in (a,b)$, такой, что $f(c) = 0$

		      $f$ принимает на разных концах, разных знаков, на левом отрцательное на правом положительное. Обозначим их $A_1,B_1$. Делим промежуток пополам, либо значение в середине ноль, либо делим снова и так далее.
		\item
		      Пусть $f(a)<f(b)$,  $f(a) < c < f(b)$
		      Рассмотрим новую функцию,  $g(x) = f(x) - C$
		      \[
			      g(a) = f(a) -  C < 0
			      .\]
		      \[
			      g(b)  = f(b) - C  >0
			      .\]
		      По пункту 1, $\exists  c ~a < c < b~ g(c) = 0 f(c) = c$

	\end{enumerate}
\end{proof}
\section{Теорема Вейрештрасса}
\begin{theorem}[о наименьшем и наибольшем значении.]
	Пусть функции $f$ задана на ограниченном замкнутом промежутке и непрерывна, тогда функция принимает на этом промежутке наибольшее и наименьшее значение.
\end{theorem}
\begin{proof}
	Пусть f не ограниченна сверху, тогда $\forall  n \exists  x_n f(x_n)> n$. Получили последовательность $x_1,x_2,x_3,\dots$ ограничена, из любой ограниченной можно выбрать сходящуюся подпоследовательность ,
	ее предел равен C. Рассматриваем последовательность  $f(x_1),f(x_2),f(x_3) \to C$.
	\[
		n_1 < n_2 < \dots
		.\]
	\[
		f(x_1) > n_1
		.\]
	\[
		f(x_2) > n_2
		.\]
	Докажем, что у $f$ есть наибольшее значение,  $f$ ограничено сверху. Пусть  $C = \sup{\{f(x) \mid x \in [a;b]\}}$. Нужно доказать  $\exists  x_0 \in [a;b]$, такое что $f(x_0) = C$

	Рассмотрим функцию  $g(x) = \frac{1}{C - f(x)}$. Она непрерывная на отрезке $[a;b]$. $\exists  M \forall  x
		\in [a,b] g(x) < M$
	\[
		\frac{1}{C - f(x)} < M
		.\]
	\[
		f(x) < C - \frac{1}{M}
		.\]
	Противоречие, так как  C супремум.
\end{proof}
\section{Дифференциальное исчисление.}
\subsection{Определение касательной.}
\begin{definition}[Производная]
	\[
		f : D \to\mathbb{R}
		.\]
	\[
		a \in D
		.\]
	Рассмотрим функцию, заданую на $D \backslash \{a\}$
	\[
		f'(a) = \lim_{x \to a} \frac{f(x) -  f(a)}{x - a}
		.\]
\end{definition}
Касательной к графику $f$ в точке с абциссой $a$ называется прямая, проходящая через эту точку и имеющия коээфициет $f'(a)$
\subsection{Производная}
\[
	x - a = h
	.\]
\[
	x = h + a
	.\]
\[
	f'(a) = \lim_{h \to 0} \frac{f(a + h) - f(a)}{h}
	.\]
\[
	f'(x) = \lim_{\varDelta x \to 0} \frac{f(x  + \varDelta x) - f(x)}{\varDelta x}
	.\]
\section{Дифференцируемсть}
\begin{definition}[Дифферинцируемая функция]
	\[
		f : D \to \mathbb{R},  a \in D
		.\]
	$f$ называют дифференцируемой в точке  $a$, если  $\exists  C$ и функция $\alpha$, заданная в некоторой окрестности нуля. $\alpha$ задана на множестве все чисел  $h$ таких что  $a + h \in D$ и $\lim_{h \to 0} \alpha(h) = 0$
	\[
		f(a + h ) - f(a) = ch + \alpha(h)h
		.\]
\end{definition}
\begin{theorem}
	\[
		f : D \to \mathbb{R}, a\in D
		.\]
	Тогда $f$ дифференцируема в точке а  $\iff$  $f$ имеет в точке а конечную производную.
\end{theorem}
\begin{proof}
	\begin{enumerate}
		\item $\rightarrow$
		      \[
			      f(a + h) - f(a) = ch + \alpha(h)h
			      .\]
		      \[
			      \lim_{h \to 0} \frac{f(a + h) + f(a)}{h} = \lim_{h \to 0} (C + \alpha(h)) =  C
			      .\]
		\item $\leftarrow$
		      Пусть $а$ имеет в точке a конечную производную
		      \[
			      \lim_{h \to 0} \frac{f(a + h) - f(a)}{h} =  f'(a)
			      .\]
		      \[
			      \frac{f(a + h) - f(a)}{h} - f'(a) = \alpha(h)
			      .\]
		      \[
			      \alpha(0) = 0
			      .\]
		      Можно считать, что равна чему угодно.
		      \[
			      \frac{f(a + h) - f(a)}{h} = f'(a) + \alpha(h)
			      .\]
		      \[
			      f(a + h) - f(a) = hf'(a) + \alpha(a)h
			      .\]
	\end{enumerate}
\end{proof}
\[
	a + h = x
	.\]
\[
	f(x) - f(a) = c(x - a) + \alpha(x - a)(x  - a)
	.\]
\[
	\alpha(x - a) =  \beta(x)
	.\]
\[
	\lim_{x \to a} \beta(x) = 0
	.\]
\[
	f(x) - f(a) = c(x - a) + \beta(x)(x  - a)= (x - a)(c + \beta(x))
	.\]
\[
	c + \beta(x) = \phi(x)
	.\]
\[
	f(x) - f(a) = \phi{(x)}(x-a)
	.\]
\[
	\beta(a) = \alpha(0) = 0
	.\]
\[
	\phi(a) =  c
	.\]
\[
	\lim_{x \to a} \phi(x) = c
	.\]
\[
	\lim_{x \to a} \phi(x) = \phi(a)
	.\]
Мы доказали, что если а дифференцируема в точке а, то существует функция, которая непрерывна в точке а, такая, что равенство
\[
	f(x) - f(a) = \phi(x)(x - a)
	.\]
\[
	\phi(a) = f'(a)
	.\]
\subsection{Упражнение}
Пусть существует $\phi$, заданаая на множестве  $D$, такая что  $\phi$ непрерывна в точке а и имеет место равенство  $f(x)  - f(a) = \phi(x)(x - a)$ и пусть  $\phi(a) =  c$. Тогда  $f$ дифференцируема в точке а и  $f'(a) = c$
\[
	\lim_{x \to a} \frac{f(x) - f(a)}{x - a} = \lim_{x \to a} \phi(x) = \phi(a) = c
	.\]
\subsection{}
\[
	f : D \to \mathbb{R}
	.\]
Следущие условия равносильны
\begin{enumerate}
	\item $f$ имеет в точке а конечную производную  $f'(a)$
	\item  $f$ дифференцируема в точке а
	\item Существует функция  $\phi$ заданная на D, такая что  $f(x) - f(a) = \phi(x)(x - a)$.  $\phi$непрерывна
\end{enumerate}
\begin{theorem}
	\[
		f: D \to \mathbb{R},  \in D
		.\]
	Если $f$ дифференцируема в точке a, то она непрерывна.
\end{theorem}
\begin{proof}
	\[
		f(x) - f(a) = \phi(x)(x-a)
		.\]
	\[
		f(x) = f(a) + \phi(x)(x - a)
		.\]
	Константа $+$ непрерывная функция.
\end{proof}
\subsection{Производная суммы}
\begin{theorem}
	\[
		f,g: D \to \mathbb{R}, a \in D
		.\]
	$f,g$ дифференцируемы в точке  $a$. Тогда  $f + g$ дифференцируема в точке а и  $(f + g)'(a) =
		f'(a) + g'(a)$
\end{theorem}
\begin{proof}
	\[
		f(a +  h) - f(a) = f'(a)h + \alpha(h)h
		.\]
	\[
		g(a + h) - g(a) = g'(a)*h + \beta(h)h
		.\]
	\[
		(f + g)(a + h) - (f + g)(a) = (f'(a)  + g'(a))h + (\alpha(h) + \beta(h))h
		.\]
\end{proof}
\begin{proof}
	\[
		f(x) - f(a) = \phi(x)(x - a)
		.\]
	$\phi$ непрерывна в точке а.
	\[
		g(x) - g(a) = \psi(x)(x - a)
		.\]
	\[
		(f + g)(x) - (f + g)(a) = (\phi(x) + \psi(x))(x - a)
		.\]
	$f + g$ дифференцируемы в точке  $a$ и  $f'(a) + g'(a) =\phi(a)  + \psi(a) = f'(a) + f'(a)$
\end{proof}
\begin{theorem}
	\[
		(fg)'(a) = f'(a)g(x) +f(a)g'(a)
		.\]
\end{theorem}
Первым способом самим
\begin{proof}
	\[
		f(x) = f(a) + \phi(x)(x - a)
		.\]
	\[
		g(x) = g(a)  + \psi(x)(x - a)
		.\]
	\[
		(fg)(x) = f(a)g(a) + f(a)(\psi(x))(x - a) + g(a)\psi(x)(x-a) + \phi(x)\psi(x)(x - a)^2
		.\]
	\[
		(fg)(x)  - (fg)(a) = (f(a)\psi(x) + g(a)\phi(x) + \phi(x)\psi(x)(x-a))(x - a)
		.\]
\end{proof}
\begin{theorem}[О производной композиции]
	\[
		f: D \to \mathbb{R}
		.\]
	\[
		g: E \to \mathbb{R}
		.\]
	\[
		f(D) \subset E
		.\]
	Пусть $a \in D, b = f(a) \in E$. $f$ дифферинцируема в точке а,  $g$ дифферинцируема в точке b. Тогда  $g \circ f$ дифференцируема в точке a.
	\[
		(g \circ f)'(a) = f'(b)*f'(a)
		.\]
\end{theorem}
\begin{proof}
	Так $f$ дифф в точке а, то  $f(x) - f(a) = \phi(x)(x - a)$
	Так как  $g$ диф в точке а, то  $g(y) - g(b) = \psi(y)(y-b)$
	\[
		\phi(a) = f'(a)
		.\]
	\[
		\psi(b) = g'(b)
		.\]
	\[
		(g \circ f)(x) - (g \circ f)(a) = g(f(x)) - g(f(a)) =
		\psi(f(x))(f(x) - f(a))) - (\psi \circ f)(x) * \phi(x)(x - a)
		.\]
\end{proof}
\subsection{Примеры}
\begin{enumerate}
	\item
	      \[
		      f(x) = \sin{x}
		      .\]
	      \[
		      f'(x) = \cos{x}
		      .\]
	      \[
		      h(x) = \sin{3x} = f(g(x))
		      .\]
	      \[
		      g(x) = 3x
		      .\]
	      \[
		      h'(x) = f'(g(x)) * g'(x)
		      .\]
	      \[
		      h' = 3\cos{3x}
		      .\]
	\item
	      \[
		      h(x) = \sin(x^2 + 5x)
		      .\]
	      \[
		      h'(x) = \cos{x^2 + 5x}(2x + 5)
		      .\]
\end{enumerate}
\subsection{Теорема о производной частного}
\begin{theorem}
	Пусть $f: D \to \mathbb{R}, g: D \to \mathbb{R}, a \in D$, $g \neq 0$ Обе функции дифференцируемы в точке а, тогда $\frac{f}{g} $ дифф в точке а и
	\[
		\Big{(}\frac{f}{g}\Big{)}' = \frac{f'(a)g(a) - f(a)f'(a)}{g^2(a)}
		.\]
\end{theorem}
\begin{proof}
	\begin{enumerate}
		\item
		      \[
			      g(x) = x, f(x) = 1
			      .\]
		      \[
			      (\frac{f}{g})'(a) = \lim_{x \to a} \frac{\frac{1}{x} - \frac{1}{a}}{x - a} = -\lim_{x \to a} \frac{1}{ax} = -\frac{1}{a^2}
			      .\]
		\item
		      \[
			      \frac{1}{g} = h \circ g
			      .\]
		      \[
			      (\frac{1}{g})'(a) = (h \circ g)(a) = h'(g(a))*g'(a)= -\frac{1}{g^2(a)} g'(a)
			      .\]
		\item
		      \[
			      \frac{f}{g} = f * \frac{1}{g}
			      .\]
	\end{enumerate}
\end{proof}
\subsection{Производная обратной функции.}
\begin{theorem}
	\[
		f : D \to \mathbb{R}
		.\]
	$f$ обратима.
	\[
		f'(a) \neq 0, b = f(a)
		.\]
	Обратная функция к $f$ непрерывна в точке b.
	\[
		f^{-1}(b) = \frac{1}{f'(a)}
		.\]

\end{theorem}
\begin{proof}
	\[
		( f^-1 )(b) = \lim_{y \to b} \frac{f^{-1}(y) - f^{-1}(b)}{y - b} =
		\lim_{x \to a} \frac{x - a}{f(x)  -f(a)} = \lim_{x \to a} \frac{1}{\frac{(f(x) - f(y))}{x - a}}  = \frac{1}{f'(a)}
		.\]
\end{proof}
\subsubsubsection{Примеры}
\[
	f(x) = \sin{x} , f = [-\frac{\pi}{2},\frac{\pi}{2}]
	.\]
\[
	(f^{-1})(b) = \frac{1}{\cos{a}} = \frac{1}{\cos{( \arcsin{b} )}} = \frac{1}{\sqrt{1 - b^2} }
	.\]
\[
	\cos^2{(\arcsin{b})} = 1 - \sin^2{\arcsin{b}} = 1 - b^2
	.\]
\[
	\cos{(\arcsin{b})} = \sqrt{1 - b^2}
	.\]
\subsection{Вычисление производных.}
\begin{enumerate}
	\item $f(x) = C $
	      \[
		      f'(a) = \lim_{x \to a} \frac{f(x) - f(a)}{x - a} = 0
		      .\]
	\item
	      $f(x) = x^\alpha$,  $x > 0, \alpha \in \mathbb{R}$
	      \[
		      f'(a) = \lim_{x \to a} \frac{f(x) - f(a)}{x -a} = \lim_{x \to a} =
		      \lim_{x \to a} = \frac{x^\alpha - a^\alpha}{x - a}
		      .\]
	      \[
		      x - a = h
		      .\]
	      \[
		      \lim_{h \to 0} \frac{(a + h)^\alpha - a^\alpha}{h} = \lim_{h \to 0} a^{\alpha} (\frac{(\frac{a + h}{a})^{\alpha} - 1}{h})
		      .\]
	      \[
		      t = \frac{h}{a}
		      .\]
	      \[
		      a^{\alpha} \lim_{t \to 0} \frac{(1 + t)^\alpha - 1}{t} * \frac{1}{a} = \alpha*a^{\alpha - 1}
		      .\]
	\item
	      \[
		      f(x) = a^x
		      .\]
	      \[
		      f'(c) = \lim_{h \to 0} \frac{f(c + h) - f(c)}{h} = \lim_{h \to 0} \frac{a^{c + h} - a^c}{h} = \lim_{h \to 0} \frac{a^c*a^h - a^c}{h}
		      .\]
	      \[
		      a^c \lim_{h \to 0} \frac{a^h - 1}{h} = a^c \ln{a}
		      .\]
	\item
	      \[
		      f(x) = \log_a{x}
		      .\]
	      \[
		      f'(x) = \lim_{h \to 0} \frac{f(x + h) - f(x)}{h} = \lim_{h \to 0} \frac{\log_a(x+h) - \log_a{x}}{h} = \frac{1}{x} \lim_{h \to 0} \frac{\log_a{( 1 + \frac{h}{x} )}}{\frac{h}{x}} = \frac{1}{x\ln{a}}
		      .\]
	\item
	      \[
		      f(x) = \ln(-x)
		      .\]
	      \[
		      f'(x) = \frac{1}{-x}*-1 = \frac{1}{x}
		      .\]
	\item
	      \[
		      f(x) = \sin{x}
		      .\]
	      \[
		      f'(x) = \lim_{h \to 0} \frac{\sin{x + h} - \sin{x}}{h} =
		      \lim_{h \to 0} \frac{2\sin{\frac{h}{2}}\cos{x + \frac{h}{2}}}{h} = \cos{x}
		      .\]
	\item
	      \[
		      f(x) = \cos{x}
		      .\]
	      \[
		      f(x) = \sin{( \frac{\pi}{2} - x )}
		      .\]
	      \[
		      f'(x) =-\cos{(\frac{ \pi}{2} - x )} = -\sin{x}
		      .\]
	\item
	      \[
		      f(x) = \tg{x}
		      .\]
	      \[
		      (\frac{\sin{x}}{\cos{x}})' = \frac{\cos^2{x} + \sin^2{x}}{\cos^{x}} = \frac{1}{\cos^2{x}}
		      .\]
	\item
	      \[
		      \ctg'{x} = -\frac{1}{\sin^2{x}}
		      .\]
              \item
                  \[
                      \arcsin'{x} = \frac{1}{\sqrt{1 - x^2} }
                  .\] 
            \item
                \[
                f'(x) = - \frac{1}{\sqrt{1 - x^2} }
                .\] 
\end{enumerate}
\section{Гиперболические функции}
\begin{enumerate}
    \item Гиперболический синус
        \[
            \sh{x} = \frac{e^x - e^{-x}}{2}
        .\] 
    \item Гиперболический косинус
        \[
            \ch{x} = \frac{e^x + e^{-x}}{2}
        .\] 
    \item Гиперболический тангенс
        \[
            \th{x} = \frac{\sh{x}}{\ch{x}} = \frac{e^x  - e^{ -x }}{e^x + e^{ -x }}
        .\] 
    \[
        \sh'{x} = \ch{x}
    .\] 
    \[
        \ch'{x}  = \sh{x}
    .\] 
\end{enumerate}
\subsection{Точки экстремума}
\begin{definition}[Точка экстреммума]
    \[
    f: D \to \mathbb{R}
    .\] 
    \[
    a \in D
    .\] 
    \begin{enumerate}
        \item 
    а называется точкой строгого минимума функции $f$, если существует окрестность $U$ точки а, лежащая в $D$, такая что  $\forall  x \in U, x\neq a, f(x) > f(a)$
\item  точкой минимума функции $f$, если  $\forall  x \in U, x\neq a, f(x) \ge  f(a)$
\item точкой строгого максиума, если $\forall x 
    \in U f(x) < f(a)$
\item 
    точка называется точкой экстренума, если она является точкой максимума или минимума.
    \item
        точка называется точкой строгого экстренума, если она являестя точкой строгого максимума или точкой строгого минимума.
    \end{enumerate}
\end{definition}
\begin{theorem}[Теорема Ферма]
    а точка экстренума функции $f$. Если  $f$ дифференцируема в точке a, то  $f'(x) = 0$
\end{theorem}
\begin{proof}
    Доказательство проведем в случае, когда а, точка максимума. $\exists U$ окрестность точки а, такое, что $\forall  x  \in U$ $f(x) \le  f(a)$.
     \[
    \lim_{x \to a} \frac{f(x)  -f(a)}{x  -a}
    .\] 
    \[
    f(x) - f(a) \le  0
    .\] 
    Если $x > a$
     \[
         \lim_{x \to a^+} \frac{f(x) -f(a)}{x - a} \le  0
    .\] 
    \[
    \lim_{x \to a^-} \frac{f(x) - f(a)}{x - a} \ge  0
    .\] 
\end{proof}
\end{document}
