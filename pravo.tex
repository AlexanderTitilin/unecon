\documentclass{article}
\usepackage[utf8]{inputenc}
\usepackage[T2A]{fontenc}
\usepackage[russian]{babel}
\usepackage{amsthm}
\usepackage{mathtools}
\usepackage{hyperref-patches}
\usepackage{hyperref}
\usepackage{listings}
\newtheorem{theorem}{Теорема}
\newtheorem{corollary}{Следствие}[theorem]
\newtheorem{lemma}[theorem]{Лемма}
\title{Лекции по праву}
\author{Александр Титилин}
\date{}
\begin{document}
\maketitle
\tableofcontents
\section{Государственно-правововое устройство Российской Федерации}
\subsection{Происхождение государства, понятие государтсва, его признаки и функции,
	форма государства, механизм государства. Понятия и признаки права, нормотивно-правой
	акт, правотворчество, правоотношения. Реализация права. Правовое и противоправное действие.
	Юридическая ответсвенность. Правосознание, правовая культура.}
\subsubsection{Методология}
Методология - набор способов, методов, инструментов. Каков метод - таков результат, при любом исследовании.
Методы называют подходами.
\subsubsection{Теории присхождения государства.}
\begin{enumerate}
	\item Теологическая. Нечто свыше создало государтво. Фома Аквинский поддерживал данную теорию.
	\item Патриархальная. Глава, отец имеет неогранниченную власть.
	\item Договорная. Люди пришли к необходимости подписания договоров о делегировании полномочий. Радищев, Руссо, Гобс.
	\item Насилия. Государство появилось, когда одни народы завоевывали других. Губович.
	\item Органическая. Государство, как организм. Платон.
	\item Психологическая. Причина возникновения государства - психология людей. Люди
	      хотят жить в месте. Питрожицкий.
	\item Иррегационная. Люди объединялись из-за сельского хозяйства.
    \item Историко-материалистическая. Общественные формации. Маркс.
    \item Другие теории.
\end{enumerate}
\subsubsection{Государство и его признаки}
Государство - это единственно возможное, всеобщее, универсальное политическая форма организации
исторически сложившегося общества, обеспечивающее решение как сугубу специальных задач,
так и общих дел вытекаюших из природы общества.

Признаки государства.
\begin{enumerate}
    \item Наличие внешних и внутренних границ.
    \item Наличие населения и граждан.
    \item Наличие публичной власти.
    \item Аппарат управления.
    \item Единое пространство (экономическое, правовое)
    \item Внещние аттрибуты (флаг, гимн, столицв, герб)
\end{enumerate}

 Функции государства.
\begin{enumerate}
    \item Принятие законов, контроль за исполнением.
    \item Регулирование прав, свобод человека, гражданина.
    \item Административно-хозяйственная.
    \item Финансовый контроль.
    \item Охрана правопорядка.
    \item Социальная функция.
    \item Экологическая.
\end{enumerate}

Внешние функции государства
\begin{enumerate}
    \item Внешнаяя политика.
    \item Подписание договоров.
    \item Национальная безопасность.
    \item Определение статуса границ и их охраны.
\end{enumerate}
\subsubsection{Механизмы государства.}

Механизмы государства - это система гос. органов, взаимосвязанная принципами, полномочиями, 
с целью управления обществом.
\subsubsection{Форма государства}
Форма государства - совокупность внешних характеристик, определяющих способ организации и устройство
государства. Включает три элемента: форма правления, форма государственного устройство, политический режим.

Форма правления.
\begin{enumerate}
    \item Монархия
    \item Республика
\end{enumerate}

Форма государственного устройства.
\begin{enumerate}
    \item Унитарное
    \item Федерация
\end{enumerate}

Политический режим
\begin{enumerate}
    \item Демократический
    \item Антидемократический.
\end{enumerate}
\end{document}
